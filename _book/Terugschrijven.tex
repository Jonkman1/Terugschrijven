\documentclass[]{book}
\usepackage{lmodern}
\usepackage{amssymb,amsmath}
\usepackage{ifxetex,ifluatex}
\usepackage{fixltx2e} % provides \textsubscript
\ifnum 0\ifxetex 1\fi\ifluatex 1\fi=0 % if pdftex
  \usepackage[T1]{fontenc}
  \usepackage[utf8]{inputenc}
\else % if luatex or xelatex
  \ifxetex
    \usepackage{mathspec}
  \else
    \usepackage{fontspec}
  \fi
  \defaultfontfeatures{Ligatures=TeX,Scale=MatchLowercase}
\fi
% use upquote if available, for straight quotes in verbatim environments
\IfFileExists{upquote.sty}{\usepackage{upquote}}{}
% use microtype if available
\IfFileExists{microtype.sty}{%
\usepackage{microtype}
\UseMicrotypeSet[protrusion]{basicmath} % disable protrusion for tt fonts
}{}
\usepackage[margin=1in]{geometry}
\usepackage{hyperref}
\hypersetup{unicode=true,
            pdftitle={TERUGSCHRIJVEN},
            pdfauthor={Harrie Jonkman},
            pdfborder={0 0 0},
            breaklinks=true}
\urlstyle{same}  % don't use monospace font for urls
\usepackage{natbib}
\bibliographystyle{apalike}
\usepackage{color}
\usepackage{fancyvrb}
\newcommand{\VerbBar}{|}
\newcommand{\VERB}{\Verb[commandchars=\\\{\}]}
\DefineVerbatimEnvironment{Highlighting}{Verbatim}{commandchars=\\\{\}}
% Add ',fontsize=\small' for more characters per line
\usepackage{framed}
\definecolor{shadecolor}{RGB}{248,248,248}
\newenvironment{Shaded}{\begin{snugshade}}{\end{snugshade}}
\newcommand{\AlertTok}[1]{\textcolor[rgb]{0.94,0.16,0.16}{#1}}
\newcommand{\AnnotationTok}[1]{\textcolor[rgb]{0.56,0.35,0.01}{\textbf{\textit{#1}}}}
\newcommand{\AttributeTok}[1]{\textcolor[rgb]{0.77,0.63,0.00}{#1}}
\newcommand{\BaseNTok}[1]{\textcolor[rgb]{0.00,0.00,0.81}{#1}}
\newcommand{\BuiltInTok}[1]{#1}
\newcommand{\CharTok}[1]{\textcolor[rgb]{0.31,0.60,0.02}{#1}}
\newcommand{\CommentTok}[1]{\textcolor[rgb]{0.56,0.35,0.01}{\textit{#1}}}
\newcommand{\CommentVarTok}[1]{\textcolor[rgb]{0.56,0.35,0.01}{\textbf{\textit{#1}}}}
\newcommand{\ConstantTok}[1]{\textcolor[rgb]{0.00,0.00,0.00}{#1}}
\newcommand{\ControlFlowTok}[1]{\textcolor[rgb]{0.13,0.29,0.53}{\textbf{#1}}}
\newcommand{\DataTypeTok}[1]{\textcolor[rgb]{0.13,0.29,0.53}{#1}}
\newcommand{\DecValTok}[1]{\textcolor[rgb]{0.00,0.00,0.81}{#1}}
\newcommand{\DocumentationTok}[1]{\textcolor[rgb]{0.56,0.35,0.01}{\textbf{\textit{#1}}}}
\newcommand{\ErrorTok}[1]{\textcolor[rgb]{0.64,0.00,0.00}{\textbf{#1}}}
\newcommand{\ExtensionTok}[1]{#1}
\newcommand{\FloatTok}[1]{\textcolor[rgb]{0.00,0.00,0.81}{#1}}
\newcommand{\FunctionTok}[1]{\textcolor[rgb]{0.00,0.00,0.00}{#1}}
\newcommand{\ImportTok}[1]{#1}
\newcommand{\InformationTok}[1]{\textcolor[rgb]{0.56,0.35,0.01}{\textbf{\textit{#1}}}}
\newcommand{\KeywordTok}[1]{\textcolor[rgb]{0.13,0.29,0.53}{\textbf{#1}}}
\newcommand{\NormalTok}[1]{#1}
\newcommand{\OperatorTok}[1]{\textcolor[rgb]{0.81,0.36,0.00}{\textbf{#1}}}
\newcommand{\OtherTok}[1]{\textcolor[rgb]{0.56,0.35,0.01}{#1}}
\newcommand{\PreprocessorTok}[1]{\textcolor[rgb]{0.56,0.35,0.01}{\textit{#1}}}
\newcommand{\RegionMarkerTok}[1]{#1}
\newcommand{\SpecialCharTok}[1]{\textcolor[rgb]{0.00,0.00,0.00}{#1}}
\newcommand{\SpecialStringTok}[1]{\textcolor[rgb]{0.31,0.60,0.02}{#1}}
\newcommand{\StringTok}[1]{\textcolor[rgb]{0.31,0.60,0.02}{#1}}
\newcommand{\VariableTok}[1]{\textcolor[rgb]{0.00,0.00,0.00}{#1}}
\newcommand{\VerbatimStringTok}[1]{\textcolor[rgb]{0.31,0.60,0.02}{#1}}
\newcommand{\WarningTok}[1]{\textcolor[rgb]{0.56,0.35,0.01}{\textbf{\textit{#1}}}}
\usepackage{longtable,booktabs}
\usepackage{graphicx,grffile}
\makeatletter
\def\maxwidth{\ifdim\Gin@nat@width>\linewidth\linewidth\else\Gin@nat@width\fi}
\def\maxheight{\ifdim\Gin@nat@height>\textheight\textheight\else\Gin@nat@height\fi}
\makeatother
% Scale images if necessary, so that they will not overflow the page
% margins by default, and it is still possible to overwrite the defaults
% using explicit options in \includegraphics[width, height, ...]{}
\setkeys{Gin}{width=\maxwidth,height=\maxheight,keepaspectratio}
\IfFileExists{parskip.sty}{%
\usepackage{parskip}
}{% else
\setlength{\parindent}{0pt}
\setlength{\parskip}{6pt plus 2pt minus 1pt}
}
\setlength{\emergencystretch}{3em}  % prevent overfull lines
\providecommand{\tightlist}{%
  \setlength{\itemsep}{0pt}\setlength{\parskip}{0pt}}
\setcounter{secnumdepth}{5}
% Redefines (sub)paragraphs to behave more like sections
\ifx\paragraph\undefined\else
\let\oldparagraph\paragraph
\renewcommand{\paragraph}[1]{\oldparagraph{#1}\mbox{}}
\fi
\ifx\subparagraph\undefined\else
\let\oldsubparagraph\subparagraph
\renewcommand{\subparagraph}[1]{\oldsubparagraph{#1}\mbox{}}
\fi

%%% Use protect on footnotes to avoid problems with footnotes in titles
\let\rmarkdownfootnote\footnote%
\def\footnote{\protect\rmarkdownfootnote}

%%% Change title format to be more compact
\usepackage{titling}

% Create subtitle command for use in maketitle
\newcommand{\subtitle}[1]{
  \posttitle{
    \begin{center}\large#1\end{center}
    }
}

\setlength{\droptitle}{-2em}

  \title{TERUGSCHRIJVEN}
    \pretitle{\vspace{\droptitle}\centering\huge}
  \posttitle{\par}
  \subtitle{Over kennis, ontwikkeling en democratie}
  \author{Harrie Jonkman}
    \preauthor{\centering\large\emph}
  \postauthor{\par}
      \predate{\centering\large\emph}
  \postdate{\par}
    \date{2018-11-15}

\usepackage{booktabs}

\begin{document}
\maketitle

{
\setcounter{tocdepth}{1}
\tableofcontents
}
\hypertarget{inleiding}{%
\chapter*{INLEIDING}\label{inleiding}}
\addcontentsline{toc}{chapter}{INLEIDING}

\begin{Shaded}
\begin{Highlighting}[]
\KeywordTok{install.packages}\NormalTok{(}\StringTok{"bookdown"}\NormalTok{)}
\CommentTok{# or the development version}
\CommentTok{# devtools::install_github("rstudio/bookdown")}
\end{Highlighting}
\end{Shaded}

\hypertarget{naar-binnen-kijken}{%
\chapter*{NAAR BINNEN KIJKEN}\label{naar-binnen-kijken}}
\addcontentsline{toc}{chapter}{NAAR BINNEN KIJKEN}

Amsterdam, juli 2017

Beste David Spiegelhalter,

Jou ken ik als bekende Engelse statisticus. Ik weet dat je een expert
bent in Bayesiaanse statistiek en waarschijnlijkheidsleer en dat jij het
medisch onderzoeksteam leidde dat het baanbrekende statistiekprogramma
WinBUGS ontwikkelde. Met dat ingenieuze programma werk ik wel eens en
het zit ongelofelijk knap in elkaar. Sinds enige jaren ben je `professor
Risk' in Cambridge en moet jij er als Winton professor voor zorgen dat
het publiek het begrip risico beter gaat begrijpen. Twee jaar geleden
werd jij voor jouw verdiensten op het terrein van statistiek geridderd.
Jij (Sir David) bent ondertussen de gekozen president van het `Engelse
Koninklijk Statistisch Genootschap'. Jij hebt jouw sporen verdiend op
het gebied van de statistiek in Engeland en ver daar buiten en je zou op
jouw lauweren kunnen rusten, benen op de tafel en een tevreden glimlach.
Onlangs bracht jij het boek \textbf{`Sex by numbers'} uit. Op de
voorkant van dit boek wordt een dichte luxaflex geopend zodat we goed
naar binnen kunnen kijken en jij kijkt met ons mee naar binnen en
vertelt ons waar we op moeten letten. Maar toen ik het boek uit had
vroeg ik mij af: waarom houd jij je nu met zo'n specialistisch onderwerp
bezig, een onderwerp dat ook nog eens zo ver van dagelijkse getallen af
lijkt te staan?

Na een paar bladzijden is het voor mij duidelijk dat jij ons overtuigend
laat zien wat de statistiek ons al niet over het seksuele gedrag van
mensen kan vertellen. Juist bij dit meest persoonlijke onderwerp wil je
zoveel mogelijk bij de feiten blijven. In het eerste hoofdstuk van het
boek laat jij zien dat de kwaliteit van het onderzoek naar seksualiteit
nogal verschillend is. Jij waardeert deze kwaliteit van de cijfers met
een nul tot en met vier-sterrensysteem. Vier sterren is dan het beste
onderzoek en onderzoek waarvan we de resultaten kunnen geloven. Denk
bijvoorbeeld aan enkele officiële statistieken zoals voor iedere 20
meisjes worden er 21 jongetjes geboren. Drie sterren krijgt onderzoek
dat redelijkerwijs te geloven is omdat er sprake is van random selectie
van deelnemers. Jij citeert veel uit het \textbf{British National Survey
of Sexual Attitudes and Lifestyles (Natsal)} en dat onderzoek krijgt
deze drie sterren. Twee sterren krijgen de onderzoeken die slechts als
ruwe schatting moeten worden beschouwd. De bekende Kinsey study, die
veel plaats krijgt in het boek (wat vooruitstrevend was Kinsey toch),
krijgt twee sterren omdat hier geen sprake was van random selectie. Dan
zijn er de onderzoeken die slechts één ster krijgen omdat de
onderliggende cijfers onbetrouwbaar zijn. Het \textbf{Hite Report},
bijvoorbeeld, krijgt van jou één ster, alleen al vanwege de lage
respons-rate van 3\% van een geselecteerde groep. Cijfers met één ster
kunnen overigens wel een hoge maatschappelijke impact hebben (zoals het
Hite-report ook laat zien). Tot slot zijn er de onderzoeken met nul
sterren waarvan de cijfers gewoon verzonnen zijn. Vervolgens laat jij in
de rest van het boek heel veel cijfers de revue passeren die voor de
seksuologie van belang zijn. Je kunt het zo gek niet bedenken en jij
laat daarbij ogenschijnlijk geen bron onbenut. Een willekeurige en
onsamenhangende greep uit die verschillende onderzoeken: 23\% van
Britten tussen de 25-34 jaar zegt dat ze de laatste vier weken geen seks
hebben gehad met het andere geslacht; mensen in Los Angeles hebben
gemiddeld 130 keer seks ieder jaar; 999 van de 1000 seksactiviteiten met
het andere geslacht leidt niet tot conceptie; 64\% van de Britse vrouwen
tussen 45 tot en met 54 jaar had vaginale seks in de laatste vier weken;
16\% van de Britse vrouwen tussen 25 tot en met 34 jaar had anale seks
het afgelopen jaar; 1,2 miljoen Britten is het geschatte aantal dat
zichzelf ziet als homoseksueel/lesbisch of biseksueel; 6\% van de Ierse
60-plussers zegt dat ze nooit een seksuele partner hebben gehad; 45\%
van de mannen wenst zichzelf een grotere penis toe; in Engeland wordt
iedere minuut 5 liter sperma geproduceerd. De cijfers duizelen mij om de
oren en ik krijg de indruk dat we met elkaar niet stilzitten. Jij laat
de cijfers ook steeds vanuit hele verschillende perspectieven zien. Zo
tel je hele verschillende activiteiten, laat jij zien wat we met het
andere geslacht doen en wat met hetzelfde geslacht, jij laat zien wat we
met onszelf doen, wat seks inhoudt voor en na het krijgen van baby's en
ga jij in op de attitudes en verlangens van mensen. Jij hebt niet alleen
oog voor het plezier maar ook voor de problemen en gaat in op de moderne
en donkere kanten van seksualiteit. En voortdurend weeg jij dan die
onderzoeken met jouw sterren systeem.

Jouw boek is een emanciperend boek. Van seksualiteit, dat zoveel mensen
onzeker kan maken, maak jij een gewoon onderwerp. Zo laat jij zien dat
twintig jaar geleden mensen nog vijf keer per maand seks met elkaar
hadden en dat dat tien jaar geleden nog vier keer was. Vandaag de dag
ligt de i-pad tussen ons in, zijn we veel te druk met ons eigen netwerk
en lijkt het er soms op dat we liever naar seks kijken dan het zelf
doen. Gemiddeld hebben mensen nu drie keer per maand seks met elkaar.
Voor velen is het misschien goed om zicht te krijgen op gemiddelden en
variaties en wat ze delen met anderen. Door verschillende bronnen en
verschillende methoden met elkaar te vergelijken kom je volgens jou tot
de meest mogelijke valide en betrouwbare cijfers. In dat proces ben jij
bijzonder creatief en hier herken je de meester. Als lezer leerde ik en
passant ook verschillende statistische aanpakken toe te passen. Zo stel
jij je bijvoorbeeld de vraag hoe je te weten kunt komen hoeveel
prostituees er in een stad werken. Daarbij gebruik jij dezelfde
schattingstechniek die ook wordt toegepast om vast te stellen hoeveel
vissen er in een vijver zitten. Je vangt er 100, je merkt ze en gooit ze
terug. Als ze goed gemixt zijn, doe je een tweede vangst van 100 en
kijkt hoeveel daarvan zijn gemerkt. Op basis van dit percentage kun je
een goede schatting maken. Door twee verschillende registraties van
prostituees (twee `vissenvangsten') met elkaar te vergelijken, die van
een gezondheidsdienst en die van de politie, en het aantal dezelfde
namen in beide registraties te tellen, kun je een schatting maken van
hoeveel prostituees er in een stad werken. Slim. Jij bent en blijft, ook
in dit boek, een statisticus die de cijfers laat zien en niet zozeer die
interpreteert of de gegevens koppelt naar moraal, onderliggende
mechanismen of hypothesen. In het begin vond ik het boek bijzonder
boeiend, maar op een gegeven moment werden al die cijfers mij toch wat
te veel. Aan het einde van het boek was ik dan toch weer blij dat ik je
boek heb uitgelezen, ook omdat jij de zaken zo goed op rij zet en je er
veel van leert. \textbf{`Sex by numbers'} is bovendien prettig
geschreven en met veel humor. Het boek is ook nog eens zeer up to date.
Jij gaat in op tegenwoordige zaken als Tinder, Vijftig Tinten Grijs,
Sexting en jij gebruikt de DSM5. Jouw boek is vooral op Engeland en
Amerika gericht. Jij gaat regelmatig in op het Engelse Natsal-onderzoek
waar jij veel waardering voor hebt. Maar de Engelse (en Amerikaanse
situatie) is niet altijd representatief voor Nederland of andere delen
van de wereld. Daar zit een beperking. \textbf{`Sex by numbers'} is
interessant voor onderzoekers en een must voor mensen die onderwijs
geven of de media te woord staan in de breedte van de seksuologie. Jij
laat ons op een gepaste manier naar binnen kijken. Na het lezen van jouw
boek kijk ik net wat anders om mij heen.

Dank je wel,

Grote groet, -Harrie

Spiegelhalter, D. (2015). \textbf{Sex by numbers. What statistics can
tell us about sexual behaviour}. London: Profile Books, 360 pag.,
€22,95.

\hypertarget{grote-uitvindingen-en-het-leven-in-een-speciale-eeuw}{%
\chapter*{GROTE UITVINDINGEN EN HET LEVEN IN EEN SPECIALE
EEUW}\label{grote-uitvindingen-en-het-leven-in-een-speciale-eeuw}}
\addcontentsline{toc}{chapter}{GROTE UITVINDINGEN EN HET LEVEN IN EEN
SPECIALE EEUW}

Amsterdam, augustus 2017

Beste Robert Gordon,

In 1870 was de levensstandaard van de gemiddelde Amerikaan laag. Niet
zoals in de middeleeuwen, maar toch. Die was lager dan die van de Brit
en weer hoger dan de Duitse levensstandaard toen. Het mensenleven was
kort (gemiddeld 45 jaar) en de kindersterfte bijzonder hoog. Mensen
woonden met grote gezinnen in kleine huizen, eigen groenten werden
verbouwd en tot eten gemaakt. Mensen consumeerden zo'n 3,5 euro per week
en dat ging bijna helemaal op aan eten, kleren en onderdak. 75 procent
van de mensen woonde enkele jaren na de Amerikaanse burgeroorlog nog op
het platteland en de samenleving was agrarisch opgebouwd. Indien er
sprake was van onderwijs was dat vaak niet meer dan de lagere school. De
huizen waren niet met elkaar verbonden: er waren geen sanitaire
voorzieningen en iedereen verwarmde het eigen huis op zijn eigen wijze.
Men vervoerde zich met het paard en het leven van gezinnen was vaak
afhankelijk van dat paard. 's Avonds was het donker op straat. De
gezondheidszorg was vooral homeopathisch van aard en op kruiden gericht.
Maar zo vanaf 1870 gebeurt er veel in Amerika en begint een speciale
eeuw die het leven van mensen totaal gaat veranderen. De levensstandaard
gaat als nooit tevoren omhoog en het karakter van de samenleving past
zich aan. In jouw formidabele (iets anders kan ik er niet tegen jou over
zeggen)\textbf{`The Rise and Fall of American Growth'} laat jij zien hoe
met de vele grote uitvindingen en innovaties dat leven in honderd jaar
totaal verandert. Nooit eerder was er zo'n economische ontwikkeling en
het is volgens jou ook maar zeer de vraag of dit ooit nog weer zal
gebeuren. Jij bent macro-econoom en economisch historicus en werkt aan
de Northwestern University en de ideale persoon om deze geschiedenis te
schrijven. Jij ziet de sterke economische groei als een eenmalige en
unieke gebeurtenis die aan het einde van de negentiende eeuw start,
vooral zichtbaar wordt in de periode 1920-1970 en die daarna weer
afvlakt.

Maar waarom is die eeuw zo speciaal? Volgens jou vinden er vanaf 1870 op
heel veel terreinen tegelijkertijd veranderingen plaats. Steden worden
met spoorlijnen met elkaar verbonden. Door de telegraaf kunnen kranten
op de dag zelf over nationale en internationale gebeurtenissen
rapporteren. Steden en huizen worden van elektriciteit voorzien en
worden ook op andere wijzen aan elkaar gekoppeld via gas, telefoon,
riolering en lopend water; het begin van de netwerksamenleving. Het
eetpatroon wordt aangepast en de voedselproductie verandert. Kleren
worden minder in huis gemaakt maar meer in fabrieken en vervolgens in
winkels verkocht. Het paard verdwijnt langzaam uit het straatbeeld en
maakt plaats voor motorische voertuigen. Het plezier en de verbeelding
veranderen met de komst van de fotografie, de muziek, de radio en de
film. Mensen gaan korter werken en maken bij de uitvoering van zwaar
werk gebruik van hulpmiddelen. Steeds meer mensen gaan in een stad wonen
en daar zijn in 1940 elektriciteit en water- en rioolverbindingen al
heel gewoon en bijna de helft van de huishoudens heeft daar dan al een
wasmachine en koelkast. De huizen worden kleiner en efficiënter
ingericht, krijgen een douche en binnen een wc, worden centraal verwarmd
en voorzien van elektrische apparaten. De bungalow doet zijn intrede.
Het transport en het vervoer gaat sneller en sneller. De auto komt er en
wegen worden aangelegd. Alles rondom informatie, communicatie en vermaak
verandert in deze periode. Het postsysteem wordt opgezet en
telefoonlijnen aangelegd. De levensverwachting van mensen stijgt snel,
zeker omdat het percentage kindersterfte sterk afneemt en de kennis over
gezondheid toeneemt. De invloed van voedsel, schoon water, riolering en
dergelijke op gezondheid wordt blootgelegd en het gezondheidssysteem met
ziekenhuizen en technologieën verandert. Bacteriële ziektes worden
aangepakt. Kinderarbeid neemt af en de arbeidsomstandigheden verbeteren.
Mensen kunnen hypotheken krijgen, kunnen zich tegen risico's verzekeren
en er komt sociale wetgeving. Het leven van 1870 kunnen we ons nu
nauwelijks nog echt goed voorstellen. Maar het dagelijks leven van veel
mensen in 1940 lijkt voor een groot deel op het leven zoals wij dat nu
nog kennen. Jij maakt duidelijk dat kwaliteit van het leven misschien
nog wel meer toeneemt dan de economische opbrengst. Jij maakt ook
duidelijk dat Amerika internationaal voorop loopt waar het ontdekkingen,
innovaties en vooruitgang na 1870 betreft. In de periode na de WOII gaat
alles in een economische stroomversnelling. Niet iedereen kon in het
eerste deel van de speciale eeuw aan alle ontwikkelingen meedoen; zeker
niet mensen in het landelijk zuiden, de armere mensen en grote delen van
de zwarte bevolking. De economische ontwikkeling zal zich verder
verbreden en er komen ook weer nieuwe ontwikkelingen bij waardoor alles
nog weer wat sneller gaat. De modernisering van Amerika wordt verder
voortgedreven door de auto die op grote schaal gebruikt gaat worden. Dan
komen de airconditioning, de snelwegen, het fastfood en de antibiotica.
Het commerciële luchtverkeer komt op gang en, bovenal, komt de televisie
in iedere huiskamer te staan, halverwege de jaren zeventig ook nog in
kleur. Na 1970 stagneert de groei en ontwikkelingen vinden al lang niet
meer overal plaats. De publieke gezondheidzorg (waar zoveel
gezondheidswinst mee gemaakt is) maakt na 1970 plaats voor meer
specialistische gezondheidszorg, waarvan de kosten enorm zullen
toenemen. Grote veranderingen vinden alleen plaats op het terrein van
informatie, communicatie en vermaak. De digitale media personaliseren en
fragmenteren dan alles ook nog eens vanaf de jaren negentig met de
I-pod, de computer, het internet en de diverse websites. De sociale
media verbinden mensen nog verder met elkaar. Rond het jaar 2000 is er
dan nog een kleine economische opleving te bemerken, maar die is
tijdelijk van aard.

Waarom, vraag jij je af, neemt het succes na de zeventiger jaren dan af?
De innovaties gaan dan nog wel verder door maar vinden, misschien wel
anders dan we vaak denken, op beperktere gebieden plaats. Dus wel op het
gebied van informatie, communicatie en vermaak maar veel minder op het
gebied van voedsel, kleding, elektronische apparaten, huisvesting,
transport, gezondheid, onderwijs en arbeidsomstandigheden. Dan mogen
bijvoorbeeld al die ICT en 3D-ontwikkelaars van tegenwoordig nog wel
optimistisch aankijken tegen de toekomst, jijzelf bent daar veel
gereserveerder over dan die techno-optimisten. Allicht dat jouw
pessimisme hier doorschiet en dat je deze ontwikkelingen onderschat.
Jouw verdienste zit er vooral in dat jij zichtbaar maakt dat innovatieve
kracht in de breedte zit. Die brede blik verliezen we tegenwoordig nogal
eens uit het oog. Maar hoe belangrijk innovatie voor jou ook is, er zijn
tegenwoordig andere maatschappelijke ontwikkelingen die volgens jou ook
een negatieve invloed hebben op economische groei. Zo heeft die huidige
samenleving te maken met een sterke toename van sociale en economische
ongelijkheid, ook binnen het onderwijs waar bovendien schooluitval
toeneemt. Dan zijn er nog demografische ontwikkelingen waar we de
komende tijd mee te maken hebben en die er toe leiden dat de
arbeidsparticipatie verder zal afnemen. In Amerika neemt de staatsschuld
steeds verder toe. En dan heb je nog de belabberde sociale situatie waar
bepaalde groepen aan de onderkant van de Amerikaanse samenleving mee te
maken hebben. Naast de verenging van de innovatieve kracht zorgen deze
zaken voor tegenwind in de Amerikaanse samenleving en tezamen zorgen ze
ervoor dat de economische groei afneemt. Jij doet hier nog wel een
aantal concrete voorstellen maar hier ligt niet de kracht van jouw boek.
Het boek lezen was voor mij een groot genot, waarschijnlijk ook omdat ik
verschillende veranderingen zelf heb mee gemaakt. Daar ben ik oud genoeg
voor. Ik heb de melk- en schillenboer nog met het paard zien rondrijden,
bij mijn oma moest ik nog buiten naar de wc en als jongetje moest ik in
de winter een paar keer per week kolen scheppen. Ik heb de tv en de
kleuren-tv, de centrale verwarming en de wasmachine zien komen. Begin
jaren tachtig zag ik de computer op het secretariaat binnen komen en
volg allerlei digitale ontwikkelingen van tegenwoordig. Jij maakt mij
duidelijk dat economische indicatoren zoals het bruto nationaal product
te beperkt zijn om economische ontwikkelingen te begrijpen, laat staan
de kwaliteit van leven en de levensstandaard vast te stellen en te
vergelijken. Jij laat ons, lezers, helemaal om ons heen kijken en maakt
ons vooral duidelijk dat om te zien waar je staat je moet weten waar je
vandaan komt.

Dank je wel.

Grote groet, -Harrie

Gordon, R.J. (2016). \textbf{The rise and fall of American growth. The
U.S. standard of living since the civil war}. Princeton and Oxford:
Princeton University Press, 762 pag., €47,95.

\hypertarget{de-wereld-in-een-zandkorrel}{%
\chapter*{DE WERELD IN EEN
ZANDKORREL}\label{de-wereld-in-een-zandkorrel}}
\addcontentsline{toc}{chapter}{DE WERELD IN EEN ZANDKORREL}

Amsterdam, september 2016

Beste Kritina Rizga,

In \emph{`De grote vlucht inwaarts. Essays over cultuur in een
onoverzichtelijke cultuur'} onderstreept de Groningse cultuurfilosoof
Thijs Lijster nog eens wat we allemaal wel weten en voelen: de wereld is
complex en onoverzichtelijk geworden. Over heel weinig hebben we
controle en allicht daarom sluiten we ons af van die grote en boze
buitenwereld die we als een natuurgegeven zijn gaan beschouwen. Het
enige wat we kunnen doen is naar binnen keren en vasthouden aan wat we
nog wel in eigen hand denken te hebben: de gezellige huiskamer, onze
spirituele huishouding en de lokale tradities en gewoonten. De wereld is
verinnerlijkt en de grote slachtoffers van deze beweging zijn de sociale
kritiek en de geschiedenis. Vergezichten en het geheel zijn in de ban
gedaan en hebben plaats gemaakt voor verscheidenheid en verschillen. Als
je een beetje filosoof van naam was de laatste decennia kondigde je wel
ergens het einde van aan: Foucault kondigde het einde van de mens aan,
Barthes van de auteur, Daniel Bell van de ideologie, Fukuyama van de
geschiedenis en Lyotard van het grote verhaal. Niets mocht en kon meer
allesomvattend worden beschreven. Maar een ander samenhangend licht
werpen op zaken was daarmee ook verdwenen en dat laat ook een leegte
achter. Daarom pleit Lijster in het verlengde van René Boomkens voor
sterke verhalen die dan, misschien niet meer alles en totaal omvattend
zijn, maar die wel degelijk eenheid aanbrengen in dat wat voor anderen
nog onoverzichtelijk en onsamenhangend is. Eigenlijk gaat het volgens
Lijster erom de wereld in een zandkorrel te zien, zoals William Blake
dat ooit zo mooi omschreef. Tegenover het vooruitgangsgeloof en het
pessimisme plaatst hij sterke verhalen; verhalen die kritisch zijn
tegenover het bestaande en mensen uitdagen ook zo'n houding aan te
nemen, die het denkbeeldige naar voren halen en waarbij schrijvers niet
bang zijn voor de consequenties en conflicten met het bestaande.

Aan sterke verhalen moest ik denken toen ik jouw nieuwe boek las:
\emph{`Mission high. One school, how experts tried to fail it, and the
students and teachers who made it triumph'}. In dat boek stel je de op
zich simpele vraag waarom het voor het rijkste en machtigste land van de
wereld (jouw Amerika) zo moeilijk is om goede scholen te bouwen in
iedere buurt? Je vraagt je af waarom jouw vrienden hun kinderen naar
scholen aan buitenkant van San Fransico sturen of, zelfs, om die reden
verhuizen. Is het niet mogelijk om binnen enkele maanden op die simpele
vraag een antwoord te vinden. Je bent journalist en de hoofdredacteur
van het tijdschrift waarvoor je werkt, adviseert jou om een school te
zoeken waar jij je kunt verdiepen in de levens van studenten en
docenten. Het kost jouw maanden om zo'n school te vinden en uiteindelijk
vind je zo'n school en begin je in 2009 met jouw onderzoek. Maar die
enkele maanden worden uiteindelijk toch een dikke vier jaar. De Mission
High-school is de oudste openbare voortgezet onderwijs school van San
Fransico, in 1890 opgezet. Er hebben bekende personen opgezeten zoals
Maya Angelou en Carlos Santana. De school maakt in haar bestaan de hele
Amerikaanse onderwijsgeschiedenis mee. In de beginjaren natuurlijk de
meer kindgerichte benadering waar John Dewey zich hard voor maakte. Dan
vanaf eind jaren vijftig wordt er voorzichtig een begin gemaakt met de
afbraak van de segregatie toen duidelijk werd dat de samenleving meer
talenten nodig had tot en met de toets- en afrekencultuur die er kwam
vanuit de hoop dat de toetsen van taal en rekenen zicht geven op de
toekomst van kinderen. In het jaar dat jij jouw onderzoek afrondt (2014)
zitten er nog 950 studenten op, met name zwarte en Spaanstalige
leerlingen. 75\% van deze leerlingen leeft in armoede en voor slechts
38\% van hen is Engels de eerste taal. Wanneer je met jouw onderzoek
begint scoort de Mission High school bijzonder laag op de officiële
lijstjes en behoort het tot de 5\% laagste scorende scholen van Amerika.
Daarom staat de school onder grote maatschappelijke druk, zoals zoveel
van deze Amerikaanse scholen: moet de school gesloten worden, de
directeur worden vervangen of de helft van het personeel worden
ontslagen? Maar er zijn andere lijstje waaruit een heel ander verhaal
spreekt: 84\% van de leerlingen wordt op vervolgopleidingen aangenomen,
89\% van de leerlingen vindt deze school goed, het slagingspercentage
van het zwarte deel van de studenten is 20\% hoger en van hen stroomt
14\% meer door. Het niveau, het slagings- en aanwezigheidspercentages
van leerlingen zijn gestegen en het schorsingspercentage nam sterk af.
Jij laat in jouw boek zien hoe studenten en docenten het dagelijkse
leven op zo'n school vormgeven. Daarvoor portretteert je de studente
Maria die het geweld in El Salvador ontvlucht en op de school langzaam
vertrouwen krijgt in zichzelf, leert schrijven en zich op school thuis
gaat voelen; de Chinese student George die naar Amerika komt om zijn
Engels te verbeteren en om met verschillen te leren omgaan en er dan
achter komt dat dat allemaal niet zo makkelijk is maar toch leert om
vragen te stellen, mee te doen aan discussies en werk durft te laten
zien aan medeleerlingen; Pablo uit Guatemala die er achter komt dat hij
homo is, daar een tijd mee zit maar op deze school een omgeving vindt
die hem ruimte en mogelijkheden biedt dit een plaats te geven; de
Afrikaans Amerikaanse Jesmyn die door haar moeder wordt opgevoed in de
armste en gewelddadigste buurt van San Fransico en zich weet op te
werken tot leerkracht, maar voor wie het leven uiteindelijk te veel is.
Maar in jouw boek komt ook een groep docenten aan het woord waaronder Mr
Roth, de oude activist die geschiedenis geeft en die er in slaagt om het
onderwijs op school te verbinden met waar kinderen vandaan komen;
Mr.~Hsu, de wiskunde leraar, die kinderen niet alleen leert rekenen maar
hen ook verbanden, patronen en strategieën leert zien; en Ms McKansey,
de zwarte lerares die de leerlingen leert schrijven en die haar eigen
passie overbrengt en dagelijks laat zien dat hetzelfde doel heel
verschillend kan worden bereikt. Rizga portretteert ook Mr.~Guthartez,
de directeur van de school die er in slaagt om een positieve
schoolcultuur te creëren en laat zien dat stabiliteit en continuïteit zo
belangrijk zijn voor deze kinderen met deze ingewikkelde achtergronden.

Volgens jou kan de democratie en de economie niet functioneren zonder
openbare scholen die toegankelijk en voor iedereen goed zijn. Maar het
is duidelijk dat de scholen niet goed zijn wanneer je zwart of
spaanstalig bent. Amerika moet hier beter mee om leren gaan en deze
talenten weten aan te spreken. Maar wat kan er dan volgens jou gedaan
worden? Bij goed onderwijs gaat het om enkele zaken: kritisch denken,
intrinsieke motivatie, veerkracht, zelfmanagement, vindingrijkheid en
sociale vaardigheden. Die zijn niet met statistische maten en computer
algoritmen vast te pakken. Lang dacht jij dat goed onderwijs zich moet
baseren op wetenschappelijke inzichten, op wat er in bepaalde landen
gebeurt en dat het meest te leren is van hoogscorende scholen met
vergelijkbare achtergronden. Maar nu kijk jij er anders tegen aan en
vind je dat de gestandaardiseerde wijze van toetsen leidt tot eenvormige
instructie en nauwelijks vast kan stellen wat individuele kinderen
kunnen en wat hun motivaties zijn. Goede docenten weten wat individuele
kinderen nodig hebben, wat hen interesseert en welke uitdagingen ze
nodig hebben. Scholen moeten daarom volgens jou op zoek gaan naar nieuwe
ideeën en leren van anderen. Bruikbare kennis van buiten moet worden
getoetst in het dagelijkse proces met individuele en collectieve
elementen. De grote jongens en snelle meisjes van Amerika hebben grootse
plannen met de toekomst van het Amerikaanse onderwijs. Zij hebben het
hoogste woord in de discussie. Maar in dat land, zo stel je subtiel
vast, gaan er iedere dag zo'n 50 miljoen studenten naar school onder
begeleiding van 3 miljoen leraren. Hun stemmen hoor je nauwelijks. Jij
hoopt dat hun inzichten en expertise onze verbeelding zullen versterken
en ons laten nadenken over wat die kwaliteit van onderwijs precies
inhoudt. Jouw verhaal over de Mission High is een sterk verhaal. Jouw
verhaal over de Mission High-school is zo'n zandkorrel waar een hele
wereld in is te kennen.

Dank je wel,

Grote groet, -Harrie

Rizga, K. (2015). \emph{Mission High. One school, how experts tried to
fail it, and the students and teachers who made it triumph.} New York:
Nation Books. 320 pagina's, Eur: 24,50

\hypertarget{kansen-en-klassen}{%
\chapter*{KANSEN EN KLASSEN}\label{kansen-en-klassen}}
\addcontentsline{toc}{chapter}{KANSEN EN KLASSEN}

Amsterdam, oktober 2017

Beste Paul de Beer, Maisha van Pinxteren en anderen,

Michael Young was een bekend onderwijssocioloog en sociaal entrepeneur.
Hij was nog erg jong toen hij vlak na de Tweede Wereldoorlog
\emph{`Facing the future'} schreef, het politieke programma waarmee
Labour in Engeland de Conservatieven onder leiding van Winston Churchill
zou verslaan. Het vormde de politieke basis voor dat beroemde kabinet
dat onder leiding van Attlee de Engelse verzorgingsstaat zou opbouwen en
waarmee het ook andere landen inspireerde. Michael Young zou
verschillende boeken schrijven en werkte mee aan de opbouw van talloze
Engelse en internationale organisaties waaronder bijvoorbeeld zijn eigen
Centre for Community Studies, de Engelse Consumentenbond en de Open
University. Maar bekend werd hij vooral met zijn boek \emph{`The Rise of
the Meritocracy'} uit 1958. Het is een satirisch boek waarin hij vanuit
het jaar 2033 en het perspectief van een meritocratische apostel
terugkijkt op de hele periode vanaf 1870. In die periode worden de
kansen niet meer zozeer zoals daarvoor verdeeld op basis van afkomst
maar op basis van verdienste. Verschillen tussen mensen worden op basis
van het meritocratisch principe vastgesteld (`je verdient het want jij
kunt het en hebt er hard voor gewerkt'). Het onderwijs is er vooral voor
de kinderen die het verdienen. Van de kinderen die het niet goed kunnen
of er met de pet naar gooien wordt verwacht dat ze het onderwijs zo snel
mogelijk verlaten. De wetenschap ondersteunt bij het soepel verlopen van
dat selectieprincipe en weet de verschillen op steeds jongere leeftijd
te detecteren. De competitie ligt er op een zo'n jong mogelijk moment en
daarna zijn de kaarten geschut. De sociale positie die iemand inneemt
wordt gelegitimeerd door intelligentie en inspanning. Maar met de nieuwe
verdeling van de kansen groeien klassen steeds verder uit elkaar en de
sociale situatie eindigt dan in 2033 in een revolte waarin de verteller,
zo lezen we in een voetnoot, wordt vermoord.

Ook in Nederland wordt na de Tweede Wereldoorlog expliciet afstand
genomen van het op afkomst gebaseerde onderwijssysteem en gaat het meer
en meer om de talenten van individuen. Anders dan in Michael Youngs
dystopie wordt het meritocratisch principe, zoals in vele landen, een
maatschappelijk ideaal. Over dat principe en die Nederlandse samenleving
schreven jullie (Paul, Maisha en de veertien anderen) een bijzonder
interessant boek: `Meritocratie. Op weg naar een nieuwe
klassensamenleving?'. Het meritocratisch principe kan als een doelmatig
principe worden gezien omdat zo iedereen op de plek terecht komt die bij
hem of haar past en de samenleving daarmee de mensen krijgt die het op
verschillende posities nodig heeft. Het is ook een rechtvaardig principe
omdat iedereen gelijke kansen heeft om zijn of haar talenten te
ontwikkelen. Kunnen we, zo vragen jullie je af, zestig jaar later ook
zeggen dat de kansen in Nederland eerlijker verdeeld worden en de
sociale afkomst (maar ook etniciteit en sekse) daarbij niet meer leidend
is? Verdienste is intelligentie en inspanning en is met onderzoek niet
makkelijk `te pakken'. Dat kan alleen op basis van het opleidingsniveau
van jongeren en hoe dat de onderwijsloopbaan en het maatschappelijk
succes vervolgens beïnvloedt. Dat is waar jullie naar kijken. De
maatschappij stond zelf natuurlijk ook niet stil. Zo was er in deze
jaren sprake van expansie van het onderwijs, upgrading van de
beroepenstructuur en emancipatie van de vrouw die op zich ook zelf weer
interacteren met het meritocratisch principe. Jullie kijken aan de ene
kant naar de rol van het onderwijs en de opvoeding en stellen je de
vraag of iemands persoonlijke capaciteiten veelmeer de toekomst zijn
gaan bepalen dan sociale afkomst. Hier laten jullie zien dat jongeren
steeds meer aan het onderwijs deelnemen en dat onderwijs en opleiding
voor het verwerven van een maatschappelijke positie belangrijk zijn
geworden. Dat is echter al decennia lang het geval. Wel is het zo dat
steeds meer jongeren aan het onderwijs deelnemen en dat de waarde van
het diploma is afgenomen. Hier valt verder op dat het niet zozeer meer
de sociaal economische positie van vader is die de onderwijsloopbaan
bepaalt maar meer het opleidingsniveau van de ouders. Die ouders spelen
met hun hulpbronnen en culturele capaciteiten over langere tijd een
ondersteunende en constante rol hierbij. Aan de andere kant kijken
jullie naar de vraag of de maatschappelijke positie ook daadwerkelijk
wordt bepaald door de individuele verdienste. Je mag verwachten dat in
onze moderne samenleving het opleidingsniveau belangrijker is geworden
voor de positie op de arbeidsmarkt. Dat kunnen jullie niet overduidelijk
aantonen. Veel hoger opgeleiden werken onder het niveau. Wel is het
risico op werkloosheid voor lager opgeleiden duidelijk groter. Daarnaast
zijn er andere factoren die de positie op de arbeidsmarkt bepalen, zoals
de samenstelling van het huishouden en de etnische achtergronden.
Alleenstaanden, alleenstaande ouders en niet-westerse allochtonen lopen
een hoger risico op werkloosheid, hebben een lager onderwijsniveau en
werken vaker in onzekere, flexibele arbeidscontracten. Bij de verdeling
van kansen lijken niet-cognitieve vaardigheden (sociale en
communicatieve vaardigheden) een rol te spelen. De sociale achtergrond
van de ouders lijkt, naast het opleidingsniveau, nog een steeds een
eigen en constante rol te spelen in de uiteindelijke maatschappelijke
positie van mensen. Enkele andere zaken vallen in die tegenwoordige
samenleving ook nog op. In het algemeen wordt het meritocratisch
principe door mensen als rechtvaardig ervaren. In het politieke domein
(zoals Eerste en Tweede Kamer, maar ook in gemeenteraden en onder
wethouders)zitten tegenwoordig personen die hoger zijn opgeleid. Jullie
stellen vast dat de politiek elite uit de hoge lagen is vervangen door
de universitaire geschoolde middenklasse en dat er sprake is van een
diplomademocratie. Aan de onderkant van de meritocratische samenleving,
waar de verliezers zitten zoals de langdurige werklozen, wordt het
meritocratisch principe ook onderschreven. Jullie stellen het systeem
niet ter discussie maar zoeken een excuus waarom het hen niet is gelukt.

Aan het einde van het boek vragen jullie je af wat dit betekent en wat
het beleid zou kunnen doen? Allereerst zouden, volgens jullie, de
kinderen met sociale achterstanden vanaf jonge leeftijd binnen schoolse
en buitenschoolse omgevingen op materieel, cultureel en sociaal terrein
moeten worden ondersteund. Daar waar kinderen het niet van huis uit mee
krijgen, zou er van hogerhand steun moeten worden geboden. Keuze voor
het vervolgonderwijs zou weer meer naar wat latere leeftijd moeten
worden verplaatst omdat sommige kinderen gewoon meer tijd nodig hebben.
Niet-cognitieve vaardigheden, zeker voor kinderen uit zwakkere sociale
milieus, verdienen meer aandacht. Op de arbeidsmarkt verdient levenslang
leren aandacht evenals het aanleren van niet-cognitieve vaardigheden.
Misschien, zo schrijven jullie, is het onvermijdelijk dat verschillen
samenhangen met het opleidingsniveau van kinderen. Maar die verschillen
(met name belonings- en inkomensverschillen) hoeven dan weer niet zo
groot te zijn. Die verschillen zijn ook op de arbeidsmarkt te bemerken
zoals kans op werk en de onzekerheid van werk. Op bepaalde momenten
moeten we ook weer niet te sterk de nadruk leggen op opleiding omdat
bepaalde groepen mensen niet aan de gestelde eisen kunnen voldoen. En
toen was ik weer terug bij Young zelf. Kort voor zijn dood in 2002 en
tijdens de hoogtijdagen van Tony Blair bracht Michael Young in The
Guardian in zijn artikel \emph{`Down with Meritocracy'} nogmaals in
herinnering dat zijn boek toch echt als satire en waarschuwing was
bedoeld en niet zozeer als ideaal. Hij zegt in dat artikel dat veel van
het boek meer dan veertig jaren na dato waren uitgekomen: de onderklasse
staat met lege handen en zonder perspectief en de leiders zijn rijker en
rijker geworden en lijken verder af te staan van de mensen waarvoor ze
zeggen op te komen. Bij Michael Young gaat het om het tegengaan van
ongelijkheid, versterken van lokaal beleid, betrekken van mensen bij de
politiek en het verheffen van alle jongeren. Het wij is bij hem veel
belangrijker dan het ik. Daar waar Young vooral kritisch was over de
meritocratie, leggen jullie aan het einde van het boek de nadruk op het
versterken van de meritocratie. De kritiek heeft bij jullie aandacht
maar die kritiek had veel meer expliciete aandacht verdient. Bij het
lezen van jullie lezenswaardig boek is het goed het werk in herinnering
te roepen van deze sociale gigant: Michael Young, Lord of Dartington.

Beer, P. de \& Pinxteren, M. van. (red., 2016). \emph{Meritocratie. Op
weg naar een nieuwe klassensamenleving?} Amsterdam: Amsterdam University
Press, 253 pag., €29,95.

\hypertarget{het-belang-van-misschien-en-ik-zou-denken-denken}{%
\chapter*{HET BELANG VAN MISSCHIEN EN IK ZOU DENKEN
denken}\label{het-belang-van-misschien-en-ik-zou-denken-denken}}
\addcontentsline{toc}{chapter}{HET BELANG VAN MISSCHIEN EN IK ZOU DENKEN
denken}

Amsterdam, november 2017

Beste Richard Sennett,

De solidariteit met mensen die hetzelfde zijn lijkt verder toe te nemen
en de verschillen met mensen die anders zijn lijken alsmaar toe te
nemen. Mensen in de westerse wereld trekken zich op eilanden terug,
lijken steeds meer moeite te hebben om op gelijke voet te komen met de
ander en steeds meer is er sprake van wij tegenover zij. In jouw laatste
boek, dat zojuist in een Nederlandse vertaling verscheen onder de titel
\emph{`Samen, een pleidooi voor samenwerken en solidariteit'}, ben je
heel duidelijk: in onze complexe samenleving kunnen we het ons niet
veroorloven de ander steeds meer uit de weg gaan. Problemen die dit
alles namelijk oplevert op het schoolplein, op de straat en in de media
zijn zo groot dat we (weer) moeten leren om te gaan met die ander;
wellevendheid zie jij als een maatschappelijke opdracht. Onze wereld is
overvol met materiële zaken en op dat gebied maken we grootse
ontwikkelingen door. De ontwikkelingen op het immateriële terrein,
echter, blijven daar ver bij achter. We moeten beter leren omgaan met
mensen die we misschien niet zo aardig vinden, die we niet helemaal
begrijpen en die gewoon net iets anders zijn dan wijzelf. Grote
praktische problemen waar we vandaag de dag tegenaan lopen brachten jou
ertoe om een trilogie te schrijven over de ervaringen van het
alledaagse. Jij omschrijft het zelf als een homo-faber project waarin je
ingaat op vaardigheden die we nodig hebben om het beter met elkaar te
rooien en weer meester worden over onszelf. Die trilogie begon met jouw
studie `De Ambachtsman. De mens als maker', een cultuurgeschiedenis over
het werk. Daarin houdt jij een pleidooi houdt om werk goed te doen
vanwege het werk en niet vanwege onszelf. De trilogie sluit jij straks
af met een derde boek waaraan jij op dit moment werkt. Dat boek zal gaan
over hoe steden zijn gebouwd en hoe met stedenbouw het leven van mensen
is te verbeteren. Maar nu ligt jouw tweede deel hiervan voor, `Samen,
een pleidooi voor samenwerken en solidariteit', dat het vermogen van
mensen om samen te werken wil vergroten.

Dit boek bestaat uit drie grote delen. In het eerste deel laat je zien
hoe samenwerking vorm is gegeven in met name de linkse politiek. In
Europa wilden de sterke Duitse vakbond, Kautsky, Lenin en alles wat
daarmee was verbonden solidariteit vooral van boven af organiseren en
alles was daarbij op het Grote Gelijk gericht. Het top-down denken
leidde er op een gegeven moment toe dat de top zich langzaamaan steeds
meer vervreemde van de basis. Hiertegenover is er de Amerikaanse
bottom-up benadering met nadruk op de gemeenschap en de
burgersamenleving. Hierbij denk je aan vrijwilligersorganisaties die in
arme, stedelijke gebieden arbeiders onderwijs en perspectief bieden en
aan opbouwwerkers die apathische armen in beweging proberen te krijgen
en gemeenschapsbanden weten te verbeteren. Het gaat jou hier om
bewegingen en personen die voortbouwden op het werk van Europese sociale
hervormers als Robert Owen in Wales en Charles Fourier in Frankrijk.
Voor jou is het duidelijk dat wij als soort alleen niet kunnen
overleven. Wij zijn sociale dieren en daarom moeten we samenwerken, ook
omdat de omgeving waarin we leven steeds verandert. Samenwerken doen we
door geven en nemen en vergelijken en tegen elkaar afzetten. Zo geven
wij onze ervaringen op bepaalde manieren vorm. Rituelen spelen mee bij
het omgaan met verschillen. Maar in de moderne tijd vinden er
veranderingen plaats in de cultuur, het werk en het dagelijkse leven
waardoor samenwerking minder open, minder dialogisch is geworden en
daarmee problematischer. In het tweede deel laat je dan zien hoe
samenwerking tegenwoordig wordt verzwakt. Allereerst is er de
ondermijnende invloed van ongelijkheid die kinderen die opgroeien
asocialer maakt. Gevoelens van inferioriteit worden geïnternaliseerd en
de ongelijkheid laat bepaalde groepen denken dat ze het toch niet ver
zullen schoppen op school en de samenleving. Dan zijn er ook de sociale
relaties op het werk die onze houding verder verzuren. Informele
relaties op het werk waren lange tijd sterk. Dat kwam omdat de arbeiders
respect hadden voor nette werkgevers en de werkgevers op hun beurt voor
betrouwbare werknemers. Problemen werden onderling besproken en als het
erop aan kwam werden tijdelijke en acute problemen aangepakt. Het werk
werd zo van drie kanten sociaal ondersteund en gaf vorm aan
gestandaardiseerde of informele wellevendheid. Dit werk veranderde de
laatste decennia in ons `durfkapitalisme' en onze
`kortetermijneconomie'. Van die verandering profiteerde de top wel maar
niet de gewone arbeiders, nu autoriteit, vertrouwen en samenwerking zijn
ondermijnd. Door structurele ongelijkheid en verandering op het werk
ontstond er een mens die niet goed meer kan omgaan met de complexiteit
van het systeem en die zich individueel terugtrekt. En wanneer de
sociale orde dan ook nog zwak is georganiseerd trekken mensen zich
verder in zichzelf terug. Zo krijgen we mensen die bang zijn om met de
ander om te gaan en dat zorgt voor narcisme en zelfgenoegzaamheid,
individualisme en onverschilligheid. Dat zorgt er vervolgens weer voor
dat mensen problemen van hun eigen mensen wel vanzelfsprekend vinden en
dat de problemen van de anderen jou niet zoveel kunnen schelen. Jij
schetst een donker beeld. Als samenwerking kan worden verzwakt kan deze
ook weer worden versterkt. Hier ben je weer optimistisch en daaraan
besteed jij in het derde deel aandacht. De moderne samenleving is,
volgens jou, aan flink herstel toe en daarbij heb je vooral oog voor het
herstellen van de samenwerking. Hoe maken wij ons dat weer eigen en waar
kunnen we van leren? Als musicus haal jij veel inspiratie uit de wereld
van muziek. Om daarin verder te komen is het belangrijk dat je veel
oefent, dat je naar elkaar luistert en je je openstelt voor de ander. Er
zijn meer voorbeelden waar we veel van kunnen leren wat samenwerken
betreft. Van arbeidsconsulenten, bijvoorbeeld, die de moeilijke taak
hebben om langdurige werklozen, die een teruggetrokken leven leiden,
angstig zijn of schaamte kennen, weer aan het werk te krijgen. In
situaties waarmee deze consulenten worden geconfronteerd helpt sympathie
en meevoelen niet. Het helpt wel als de emotionele temperatuur van de
werkloze wordt verlaagd en zij weer aan het werk komen.
Conflictmanagement, vergadertechnieken en professionele diplomatie zijn
andere voorbeelden waar je uit put om van samenwerken te leren. Het gaat
jou om vaardigheden waarmee grenzen worden verlegd, waarmee zorgen en
belangen worden geherformuleerd, zaken open wordt gebroken, sociale
distantie wordt gecreëerd en er een basis ontstaat om verder te gaan met
de ander.

Drie personen hebben jou als student beinvloed: David Riesman (bekend
van zijn studie \emph{`The Lonely Crowd'} waarin hij laat zien hoe het
karakter van de Amerikaan vlak na de WOII veranderd is en traditie en
autonomie plaats hebben gemaakt voor het nadoen van ander), Erik Erikson
(de moderne psycho-analyticus die over identiteit en identiteitscrises
in verschillende levensfasen schreef) en Hannah Arendt (de grote
filosofe van de twintigste eeuw die de moderne mens, het kwaad en
totalitaire systemen analyseerde). Met zulke leermeesters kan het bijna
niet meer misgaan. `Samen' doet misschien wel het meest aan Hannah
Ahrendts boek `The Human Condition' denken dat gaat over het goede leven
dat bestaat uit arbeid, werk en handelen. In de moderne tijd gaat het
volgens Arendt steeds om consumeren, produceren en onze eigen wereld en
steeds minder om de publieke zaak. Jijzelf bent minder filosofisch en
abstract en meer praktisch. Ook jou gaat het om het goede leven, dat
voor jou een leven met de ander is, en het ontwikkelen van een innerlijk
levensdoel door middel van samenwerking. Samenwerking zie jij als een
vaardigheid die we nodig hebben in deze tijd, een ambacht bijna, om
dialogische relaties aan te gaan, ruimte te geven en te luisteren naar
wat de ander wil zeggen en wat hij of zij nodig heeft. Wanneer je als
lezer het boek uit hebt, roept het boek in eerste instantie ook
irritatie op. Althans bij mij. Waarom houdt je de lijn niet meer vast,
hadden die redacteuren jou niet meer op koers kunnen houden? Dat soort
vragen. Als je het nog eens doorleest en ook jouw aanhangsel leest
(`Coda. De kat van Montaigne') begrijp je jou beter. Het gaat je om de
kunst van luisteren, op gezette tijden zwijgen en tact tonen. Net als
Montaigne, jouw grote voorbeeld, wil je de lezer op dwaalsporen zetten,
onderwerpen van onverwachte invalshoeken voorzien en er uiteindelijk een
mozaïek van fragmenten van maken dat toch één geheel vormt. Net zoals
Montaigne dat in zijn `Essays' heeft gedaan. Montaigne kijkt tegen de
ander aan zoals hij tegen zijn kat aankijkt: blij dat hij anders is en
soms verwonderd over wat hij doet. Dat wil jij ook en die
raadselachtigheid moeten we terug zien te krijgen. Nu gaat het niet
zozeer om wat jij schrijft maar vooral om de aannames die je daarbij
heeft. Als je dat als lezer doorhebt en als je zo naar jou luistert,
spreekt dit boek bijzonder aan. Ik heb het boek net op tijd gelezen om
het verkiezingsjaar 2017 aan te kunnen. Het maakt duidelijk dat
samenwerking niet vanzelf komt maar uitgebreide ervaring en oefening
vraagt. De kunst ook om misschien te zeggen en ik zou denken dat.

Sennett, R. (2016). \emph{Samen. Een pleidooi voor samenwerken en
solidariteit.} Amsterdam: Meulenhoff, 399 pag., €24,99.

\hypertarget{grit}{%
\chapter*{GRIT}\label{grit}}
\addcontentsline{toc}{chapter}{GRIT}

Amsterdam, december 2016

Beste Paul en Angela,

Zo nu en dan zijn er van die Engelse woorden die op de een of andere
wijze een eigen leven gaan leiden. Jaren geleden was er plotseling het
woord `flow' dat werd geïntroduceerd door, en nu moet ik even goed
opletten of ik het goed schrijf, Mihaly Csikszentmihalyi. Het gaat hier
om een geestelijke toestand van iemand die volledig opgaat in zijn of
haar bezigheid. De laatste jaren is er een nieuw woord dat steeds vaker
opduikt, ook weer uit de hoek van de positieve psychologie, en dat we
nog veel tegen zullen komen de komende tijd: `grit'. Let maar op. Het is
zoiets als vastberadenheid, maar het Nederlandse woord is natuurlijk
veel minder krachtig.

Paul, jij bent een Amerikaanse publicist en schreef een paar jaar
geleden \emph{`How Children Succeed: Grit, Curiosity and the Hidden
Power of Character'}. Recent kwam jouw nieuwe boek uit onder de titel
(hoe kan het anders, jij bent een Amerikaan): *`Helping Children
Succeed: What Works and Why?* Het is, schrijf je, duidelijk dat er een
grote groep kinderen qua schoolresultaten en onderwijssucces
achterblijft. Het gaat hier met name om kinderen uit de gezinnen met
lagere sociale economische achtergronden. Onderwijzers hebben er vaak
hun handen vol aan om, in ieder geval een deel van die jongeren, te
motiveren en mee te krijgen in het dagelijkse onderwijsritme. We weten
waarom bepaalde kinderen uitvallen maar veel minder goed weten we waarom
bepaalde kinderen succes hebben. Hoe wordt succes behaald en langs welke
lijnen loopt dit? Het onderzoek dat er naar gedaan is, heeft volgens jou
duidelijk gemaakt dat de link naar succes niet zozeer in de cognitieve
zaken en het leren zelf zit. Nee, het zit veel meer in niet-cognitieve
zaken zoals weerbaarheid, optimisme, zelfcontrole en (dus dat ene woord)
`grit' (dat in het Nederlands misschien nog wel het beste met
vastberadenheid is te vertalen). Het zijn deze kwaliteiten die het
verschil maken en die in het onderwijs aan kinderen uit met name de
sociaal zwakkere milieus zo essentieel zijn. Juist deze kinderen moeten
leren omgaan met tegenslag en weten hoe ze hun ervaringen naar hun hand
kunnen zetten. Juist voor deze kinderen zijn rustige, consistente en
wederkerige interacties van het grootste belang voor de toekomst. Dat
moeten ze vooral meekrijgen in hun vroege jaren aan de hand van hun
ouders en opvoeders. Ook in voorschoolse en schoolse settings kunnen ze
hun aandacht en concentratie leren te versterken en hun stress onder
controle krijgen. Juist daarom moeten ouders en leerkrachten omgevingen
vormen waar deze niet-cognitieve competenties, autonomie en
verbondenheid worden ontwikkeld, waar kinderen grip krijgen op
houdingen, percepties en representaties en daarmee op hun academische
taken en gedrag. Kinderen en jongeren moeten over een langere tijd het
gevoel krijgen dat ze erbij horen, dat hun inspanningen beloond worden,
dat ze soms succes kunnen hebben en dat wat ze doen waarde heeft.
Angela, jij bent een andere machinist van deze grit-trein die op dit
moment door Amerika dendert. Ooit was je wiskundelerares en nu werk je
als psychologe in de wetenschap. Al vroeg werd het jou duidelijk dat het
IQ van een kind niet zo belangrijk is voor het succes van het kind.
Succes heeft volgens jou veel meer met motivatie te maken en daar moet
in het onderwijs meer aandacht voor komen. Uit jouw onderzoek naar
succes in verschillende situaties kwam er één duidelijke voorspeller
uit. Niet dus de sociaal economische situatie waarin kinderen opgroeien
of de toets gegevens, nee, nee, het is ook volgens jou de `grit', wat je
definieert als passie en volharding. Oftewel, de wijze waarop je iedere
dag en over langere tijd met je toekomst bezig bent en hard werken.
Daarover weten we eigenlijk betrekkelijk weinig en het enige wat we
weten is dat het verandert met inspanning.

Het idee van de `grit' slaat enorm aan in Amerika en krijgt heel veel
aandacht op televisie en internet en in de kranten. Grote potten
onderzoeksgeld gaan naar dit onderwerp toe. Het is natuurlijk ook een
echt Amerikaans onderwerp waarin het maatschappelijk opklimmen verbonden
wordt met hard werken, koers houden, doorzettingsvermogen en
karaktervorming. Je hebt hierbij meteen die spijkerbroek en de
opgestroopte mouwen voor ogen. Dat is op zich niet zo nieuw en misschien
ook wel de basis van de Amerikaanse droom. Het perspectief dat jullie
positieve psychologen (Paul en Angela) is natuurlijk heel anders dan dat
van Robert Putnam. Putnam prikte vorige jaar die Amerikaanse droom nog
door toen hij in `Our Kids' heel duidelijk maakte dat van gelijkheid van
onderwijskansen al heel lang geen sprake meer is in Amerika. De
Amerikaanse samenleving is uit elkaar gedreven en twee groepen leven
overduidelijk gescheiden van elkaar.\\
Er is nog een andere opmerking te maken over `grit' en dat is iets waar
de onderwijsjournalist Paul Thomas ons onlangs terecht op heeft gewezen.
Hij plaatst verschillende kritische kanttekeningen bij het nieuwe
perspectief waar jullie, Paul en Angela, warm voor lopen. We moeten zo'n
perspectief niet zomaar meteen met z'n allen omarmen en we moeten er op
z'n minst en zeker in de beginfase een aantal vragen bijstellen. Grit
wordt bijvoorbeeld gepresenteerd als een karaktertrek van de hogere orde
met sterk voorspellende waarde naar zowel toekomstig maatschappelijk
succes als schoolresultaten van kinderen. Maar onderzoek naar honderden
effect-sizes in tientallen studies met tienduizenden individuen geeft
vooralsnog toch een ander beeld van die effecten. De effecten die jullie
(Paul en Angela) schetsen zijn helemaal niet zo duidelijk als jullie
overtuigd in de media naar voren brengen. De verbanden liggen toch net
wat anders. Vastberadenheid als karaktereigenschap lijkt veel meer op
iets als zorgvuldigheid en dat is nu juist een karaktereigenschap die
net wat moeilijker te veranderen is. Kortom, de wetenschappelijke basis
waarop grit gebaseerd is en waar met name jij, Angela, naar refereert,
lijkt vooralsnog dun. Daarom alleen al moeten we zo'n idee niet direct
zomaar omarmen. We moeten er naar luisteren, we moeten er van leren en
onderzoeken wat mogelijk is. Maar we moeten er vooral ook kritische
vragen over blijven stellen. Zo'n kritische houding wordt tegenwoordig
nogal eens gemist. Journalisten moeten niet bij één verhaal van een
buitengewone school blijven steken, ze moeten wetenschappelijk onderzoek
niet te simpel opvatten, ze moeten historische patronen herkennen, niet
te optimistisch zijn over de mogelijkheden van het onderwijs, de invloed
van de leraar niet overdrijven, niet alleen naar het individuele maar
ook naar de maatschappelijke context kijken, mythen doorprikken,
onderwijzers meer zelf aan het woord laten en de verbeelding laten
spreken. Van journalisten mag en moet je vooral kritiek verwachten. Ook
tegenover zo'n nieuw begrip als grit dat aantrekkelijk lijkt en waar
zoveel mensen mee weglopen. Allicht moeten we vastberaden zijn om
kinderen vastberaden te maken en daar van alles voor doen. Ook in
Nederland. Maar we moeten ook vastberaden zijn en blijven om de goede
vragen te blijven stellen, ook aan jullie, Paul en Angela.

Dank jullie wel.

Grote groet, -Harrie

Tough, P. (2012). \emph{How Children Succeed: Grit, Curiosity, and the
Hidden Power of Character.} New York: Houghton Mofflin Harcourt
Publishing Company. pag., €,.

Tough, P. (2016). \emph{Helping Children Succeed: What Works and Why.}
New York: Houghton Mofflin Harcourt Publishing Company. pag., €,.

Duckworth, A. (2016). Grit. The Power of Passion and Perseverance. New
York: Scribner. pag., €,.

\hypertarget{de-ouder-als-meubelmaker-of-tuinman}{%
\chapter*{DE OUDER ALS MEUBELMAKER OF
TUINMAN}\label{de-ouder-als-meubelmaker-of-tuinman}}
\addcontentsline{toc}{chapter}{DE OUDER ALS MEUBELMAKER OF TUINMAN}

Amsterdam, januari 2017

Beste Alison Gopnik,

Jij moet een moedig persoon zijn. Over opvoeden zijn er alleen al bij
Amazon 60.000 titels en toch heb je onlangs over dat onderwerp een boek
geschreven. Je bent bekend geworden met boeken als \emph{`The scientist
in the crib. What early learning tells us about the mind'} en \emph{`The
Philosophical baby: what children's minds tells us about truth, love,
and the meaning of life'}. In die boeken laat je zien dat er grote
overeenkomsten zijn tussen hoe wetenschappers en hoe jonge kinderen
denken. In jouw laatste boek schrijf je over opvoeden en de rol van
ouders daarbij. Misschien is dat boek niet zozeer vernieuwend, maar het
zet je wel aan het denken. Ik heb het hier over \emph{`The gardener and
the carpenter. What the new science of child development tells us about
the relationship between parents and children'}. Jij bent in dat boek
zeer kritisch over hoe ouders tegenwoordig kinderen opvoeden en hoe ze
erover denken. `Parenting' heet het in het Engels en dat is net wat
specifieker dan opvoeden zoals wij het in Nederland noemen. Dat
`parenting' of opvoeden door ouders is volgens jou tegenwoordig een
doelgerichte activiteit geworden, een soort werk dat kinderen beter,
gelukkiger en succesvoller wil maken. Opvoeden is al lang niet meer
alleen wat we doen met kinderen maar opvoeden schrijft vooral voor wat
er moet gebeuren. Die doelgerichte benadering leidt er volgens jou toe
dat we onze kinderen eindeloos met kinderen van onze vrienden
vergelijken, kijken naar wat onze kinderen op school presteren en dat
vergelijken met hoe andere kinderen het doen. We zijn niet met het nu
maar vooral met de toekomst bezig. Als ouder willen we de laatste
technieken toepassen en volgen we de aanwijzingen van diverse experts.
We zijn alleen maar tevreden als we een goed uitgewerkt opvoedingsmodel
kunnen toepassen en als onze opvoeding een bepaald resultaat oplevert.

Jijzelf groeide op in de zestiger jaren, die gelukkige vijf minuten,
zoals je dat zo mooi omschrijft, tussen de pil en AIDS. Jij maakte deel
uit van die kritische generatie die nogal wat had op te merken over
ouders die zelf vaak in de crisistijd waren opgevoed. Nu stel je, onder
tussen zelf moeder en grootmoeder, dat jouw generatie (onze generatie)
het er niet heel veel beter vanaf heeft gebracht. Volgens jou is de
manier waarop we tegen opvoeden aankijken een verkeerde manier omdat die
niet past bij wetenschappelijke inzichten die laten zien dat we het de
kinderen veel meer op hun eigen manier moeten laten doen. Je hebt gelijk
dat het ouders van tegenwoordig veel te weinig gaat om variatie, risico
en innovatie en dat de opvoeding op die manier misschien wel te weinig
aansluit bij het evolutionaire doel van de kindertijd. Kindertijd is
vooral de tijd van exploratie, nieuwsgierigheid en spel en dat heb je
nodig voorafgaand aan de fase van exploiteren, verantwoordelijkheid en
werk. Wat er van elk kind wordt is onvoorspelbaar en uniek en het
resultaat van vreemde combinaties van genen, ervaringen, cultuur en
geluk. We moeten niet te snel een kind willen maken maar veelmeer
liefde, veiligheid en stabiliteit bieden waarin kinderen op hun manier
kunnen groeien. We kunnen nu één keer niet kinderen lerend maken, maar
we kunnen ze wel laten leren. Ouderschap zie jij terecht als een
belangrijk deel van de levenscyclus waarbij onze ouders ons het verleden
gaven en wij op onze beurt de toekomst aan onze kinderen doorgeven. Meer
dan bij welk ander levend soort ook zijn kinderen bij ons mensen lange
tijd afhankelijk van opvoeders. Zelfs in vroegere tijden waren ze niet
voor hun vijftiende zelfstandig. Kinderen waren daarbij niet alleen
afhankelijk van hun eigen ouders maar ook van `allo-ouders', het netwerk
van grootouders, ooms, tantes, neven, nichten en vrienden. Alleen komen
we als soort nergens en we ontwikkelen onszelf alleen in zo'n netwerk
van zorg en liefde. Lange tijd groeiden kinderen op in uitgebreide
families maar vandaag de dag is dat netwerk veel kleiner geworden. De
opvoedingstaak ligt nu meer dan ooit bij de ouders, die zelf lange tijd
naar school zijn geweest en al enige tijd werken voordat ze ouder zijn
geworden. Op school en werk hebben ze zich dat doelgerichte, met de
nadruk op kennis en competenties, eigen gemaakt. Op school en werk leidt
dat natuurlijk tot succes. Maar in de opvoeding thuis is dat niet zozeer
het geval. Ook lijkt het er op dat ouders van nu alleen maar oog voor
het detail hebben. Ze stellen zichzelf vragen als hoe lang moeten ze hun
kinderen moeten laten huilen, of de computer wel goed voor ze is en of
ze wel lang genoeg huiswerk maken. De grote lijn lijken ze uit het oog
te zijn verloren.

Van jouw boek heb ik veel geleerd. Als ontwikkelingspsycholoog bied je
de lezer nieuwe inzichten, als filosoof lever je interessante dilemma's
en als ouder/grootouder laat je zien met welke concrete situaties je te
maken hebt gehad. Opvoeden tegenwoordig, zo schrijf jij, lijkt misschien
wel het meest op het werk van een meubelmaker. Er is ruim aandacht voor
het juiste materiaal waarmee gewerkt wordt, voor de juiste bewerking en
ouders willen net als de meubelmaker ervoor zorgen dat het juiste
product wordt afgemaakt. In het proces gaat het om precisie en controle
en chaos en verschil moeten zoveel mogelijk worden voorkomen. Maar
jijzelf hebt veel op met het idee dat opvoeden als het zorgen voor een
tuin is. Want als tuinman moet je de tuin beschermen en planten ruimte
geven waar nodig. Je moet er veel voor doen maar het onverwachte is hier
veel belangrijker. Net als de tuinman kun je als ouder niet alles
voorspellen en variatie en wanorde lijken er hier wel degelijk toe te
doen. Ik moet je eerlijk zeggen dat ik samen met mijn vrouw vaak in de
tuin werk. Ik weet (en mijn vrouw nog veel beter) dat tuinieren wel
degelijk doelgerichte kanten kent. Je moet soms dingen doen om een half
jaar later bepaalde resultaten te bereiken. Niet alles maar wel voor een
groot deel. Bovendien heb je hele verschillende tuinen. Natuurlijke
tuinen en tuinen waar precisie en controle wel van afspatten. Je hebt
Engelse tuinen en je hebt Franse tuinen. Kinderen waarvan de
ontwikkeling in gevaar komt om welke redenen dan ook, zullen ondersteund
moeten worden. Over hoe dat het beste kan en waar we rekening mee moeten
houden moeten we na blijven denken. Dan doen we ook in de tuin als
planten niet tot hun recht komen. Dat zullen we doelgericht moeten
blijven doen en ik neem aan dat je die gedachte deelt. Met het
vergelijken van opvoeden met houtbewerken aan de ene kant en tuinieren
aan de andere kant kan ik niet helemaal met jou meekomen. Misschien,
denk ik zelf, had je voor jouw boek veel beter de metafoor van het lange
en avontuurlijke reizen met het aankomen als doel kunnen nemen, zoals
Kavafis dat zo treffend in zijn beroemde gedicht `Ithaka' beschrijft. De
reis die vele jaren duurt maar rijk is aan wat je onderweg beleeft. Bij
opvoeden gaat het om het aankomen maar veel meer nog om die mooie reis.
Maar laat ik zelf ook de grote lijn van jouw boek in de gaten houden en
het punt dat je maakt is terecht.

Dank je wel,

Grote groet, -Harrie

Gopnik, A.(2016). \emph{The gardener and the carpenter. What the new
science of child development tells us about the relationship between
parents and children.} New York: Farrar, Strauss and Giroux, 320 pag.,
€22,16.

\hypertarget{grenzen-verleggen}{%
\chapter*{GRENZEN VERLEGGEN}\label{grenzen-verleggen}}
\addcontentsline{toc}{chapter}{GRENZEN VERLEGGEN}

Amsterdam, februari 2017

Beste Julian Baggini,

Wij zijn ons verstand verloren en met die hartenkreet val je in jouw
nieuwe boek meteen met de deur in huis. Was ons rede en rationaliteit
meer dan 2000 jaar bijzonder dierbaar, tegenwoordig lijken die er minder
toe te doen. Verschillende politici stellen openlijk de vraag of experts
nog wel zo nodig zijn. Feiten zijn minder belangrijk dan emoties. Wij
hebben, zo lijkt het, steeds minder vertrouwen in de rede. Wat maakt het
allemaal nog uit. Het kapitalisme heeft veel van de menselijkheid
weggenomen, we hebben Auschwitz gehad, veel wordt bepaald door onze
genen en de grote multinationals doen toch wat ze niet laten kunnen. We
volgen tegenwoordig ons hart en onze passie omdat we denken dat we daar
nog greep op hebben. Over dat irrationele landschap maak jij je nogal
zorgen. Jij vindt jezelf een geschikt persoon om hierover een boek te
schrijven. Ik geloof dat je gelijk hebt. Je bent universitair geschoold
en staat daarmee met één been in de wetenschap. Maar de standaard
academische stijl die is jou te `high brow', te precies en niet zelden
rigide. Je voelt je de laatste jaren thuis buiten de universiteit waar
je werkt als redacteur, journalist en schrijver. Zo sta je met het
andere been in de samenleving. Je hebt de afgelopen jaren honderden
wetenschappers geïnterviewd en weet maar al te goed als buitenstaander
wat er in die wereld speelt. Je hebt je bezig gehouden met filosofie en
religie. Jij overziet, zeg je zelf, het hele bos en niet, zoals
wetenschappers zelf vaak, een boom, een enkele struik of zelfs een blad.
In \emph{`The edge of reason. A rational skeptic in an irrational
world'} vraag jij je af hoe het allemaal zo ver heeft kunnen komen met
de rede en wat er moet gebeuren om die weer betekenis te geven.

Sinds Plato is de rede altijd het hoogst haalbare geweest. De wereld is
veel complexer geworden maar volgens jou kennen rede en rationaliteit
vier uitgangspunten die terug te voeren zijn op Plato. Elk uitgangspunt
zie jij als een mythe die volgens jou een voor een moet worden
doorgeprikt. De eerste van de vier mythes, waar we nog steeds last van
hebben, is de mythe dat subjectiviteit er in het vormen van een oordeel
er niet toe doet. Rede zou eigenlijk altijd tot een bepaalde conclusie
moeten leiden. Maar jij laat zien dat er altijd een subjectief element
meespeelt. Op het hoogste niveau van de natuurwetenschap, bijvoorbeeld,
is een klein groepje topwetenschappers het helemaal niet met elkaar eens
over de centraal onderliggende theorie en kunnen niet tot een vergelijk
komen hoe het in elkaar zit. Ook de tweede mythe is al duizenden jaren
oud en houdt in dat de ziel zou worden voortgetrokken door twee paarden:
het intellect of rede aan de ene kant en de emotie aan de andere kant.
Als de rede maar de touwtjes in handen heeft, komt het goed. Maar die
rede kan helemaal niet functioneren zonder die emotie en die onderlinge
verbondenheid moet ook erkend worden. Het was ook Plato die met het idee
kwam dat rationaliteit en rede ons op een goede manier aanzet tot
handelen en het dan vanzelf wel tot het goede handelen leidt. Tot slot
zou het volgens diezelfde goede, oude Plato goed zijn als in
gemeenschappen filosofen koningen zouden zijn. Zover is het nooit
gekomen maar de geschiedenis heeft wel overduidelijk gemaakt wat het
betekent als er teveel vertrouwen is in objectieve waarheid en als
systemen menen te weten hoe mensen zouden moeten leven. De geschiedenis
heeft duidelijk gemaakt dat een rationele staat veel meer subtiliteit
veronderstelt.\\
Misschien moeten we wel erkennen dat de rede van de troon is gestoten
juist omdat haar ster zo hoog was gerezen. Maar nu die zo in twijfel
wordt getrokken, moet de positie ervan worden geherdefinieerd en opnieuw
bezien. Die herdefiniëring is nodig omdat de rede en rationaliteit zijn
gestrand in het mulle zand van populisme en postmodernisme. Om de kar
weer aan het rijden te krijgen moeten we meer los komen van de harde,
steriele wetenschappelijke kijk op de zaak die zo de boventoon die zo de
boventoon voert. Volgens jou is het belangrijk dat de grenzen van de
rede opnieuw worden vastgesteld en we meer oog hebben voor kritisch
denken waar denken nodig is. Jijzelf plaatst diverse kanttekeningen bij
rede en rationaliteit en je vraagt je inderdaad als lezer af, wat heb
jijzelf nog met die rede: ben je niet veel te kritisch om die nog te
kunnen verdedigen? Maar jij kijkt daar juist anders tegen aan. Juist
omdat je zoveel met rede en rationaliteit op hebt, wil je snappen wat er
scheelt. Jij vergelijkt jezelf met een topatleet. Die moet ook heel veel
trainen en er voor zorgen dat zijn zwakke plekken aandacht krijgen.
Volgens jou is het nodig dat intellectuelen en academici zelf ook nagaan
denken over waar het bij rede en rationaliteit om moet gaan en welke
vorm bij deze tijd past. Jij staat een sceptische houding ten opzicht
van de rede voor en een bredere benadering dan die waar nu sprake van
is. Het gaat jou om duidelijke en vanzelfsprekende principes, om
duidelijke stappen die gezet kunnen worden, om het regelmatig herbezien
van conclusies en het steeds onder ogen zien van de consequenties ervan.
Dat betekent volgens jou per definitie dat we er heel langzaam op
vooruit gaan, dat totale waarheid misschien wel nooit wordt bereikt,
maar wel dat er een soort stabiliteit en zekerheid nodig is en dat in de
basis duidelijk is waar we voor staan. Jij voelt je in deze zin erg
thuis in de gedachtewereld van Hume, die ook vindt dat we de beperkingen
van de rede moeten accepteren en dat er daarbij altijd een bepaalde
graad van twijfel aanwezig moet zijn. Jij vindt dat we weer geloof
moeten krijgen in de kracht en de reikwijdte van rationaliteit. Maar ook
dat academici en intellectuelen zich weer verantwoordelijk gaan voelen
voor dat rationele domein en ervoor zorgen dat we hierbinnen weer met
elkaar kunnen discussiëren en argumenteren. Verschillen zullen en moeten
er ook blijven als we het maar eens zijn over de gedeelde basis van de
rationaliteit en die basis zelf niet teveel in twijfel trekken.

In jouw boek heb je vooral ook voor het proces en hoe we met elkaar
omgaan. Je geeft er zelfs een hele gids met 50 tips bij, die voor mij
niet hadden gehoeven. Maar het belangrijkste voor jou is dat we weer
kunnen schaatsen met elkaar ook al is het ijs dun. Wij hebben geen
andere keuze. Als ik naar buiten kijk zie ik dat er vandaag een dun
laagje ijs ligt op het water waar mijn woning ligt. Wat lijkt mij dat
weer heerlijk, schaatsen. Misschien zijn we het met z'n allen wel een
beetje verleerd. Met enkele aanwijzingen, zoals jij die in dit mooie
boek geeft, moet dat toch weer lukken.

Dank je wel.

Grote groet, -Harrie

Baggini, J.(2016). \emph{The edge of reason. A rational skeptic in an
irrational world.} New Heaven and London: Yale University Press. €18,10

Jijzelf groeide op in de zestiger jaren, die gelukkige vijf minuten,
zoals je dat zo mooi omschrijft, tussen de pil en AIDS. Jij maakte deel
uit van die kritische generatie die nogal wat had op te merken over
ouders die zelf vaak in de crisistijd waren opgevoed. Nu stel je, onder
tussen zelf moeder en grootmoeder, dat jouw generatie (onze generatie)
het er niet heel veel beter vanaf heeft gebracht. Volgens jou is de
manier waarop we tegen opvoeden aankijken een verkeerde manier omdat die
niet past bij wetenschappelijke inzichten die laten zien dat we het de
kinderen veel meer op hun eigen manier moeten laten doen. Je hebt gelijk
dat het ouders van tegenwoordig veel te weinig gaat om variatie, risico
en innovatie en dat de opvoeding op die manier misschien wel te weinig
aansluit bij het evolutionaire doel van de kindertijd. Kindertijd is
vooral de tijd van exploratie, nieuwsgierigheid en spel en dat heb je
nodig voorafgaand aan de fase van exploiteren, verantwoordelijkheid en
werk. Wat er van elk kind wordt is onvoorspelbaar en uniek en het
resultaat van vreemde combinaties van genen, ervaringen, cultuur en
geluk. We moeten niet te snel een kind willen maken maar veelmeer
liefde, veiligheid en stabiliteit bieden waarin kinderen op hun manier
kunnen groeien. We kunnen nu één keer niet kinderen lerend maken, maar
we kunnen ze wel laten leren. Ouderschap zie jij terecht als een
belangrijk deel van de levenscyclus waarbij onze ouders ons het verleden
gaven en wij op onze beurt de toekomst aan onze kinderen doorgeven. Meer
dan bij welk ander levend soort ook zijn kinderen bij ons mensen lange
tijd afhankelijk van opvoeders. Zelfs in vroegere tijden waren ze niet
voor hun vijftiende zelfstandig. Kinderen waren daarbij niet alleen
afhankelijk van hun eigen ouders maar ook van `allo-ouders', het netwerk
van grootouders, ooms, tantes, neven, nichten en vrienden. Alleen komen
we als soort nergens en we ontwikkelen onszelf alleen in zo'n netwerk
van zorg en liefde. Lange tijd groeiden kinderen op in uitgebreide
families maar vandaag de dag is dat netwerk veel kleiner geworden. De
opvoedingstaak ligt nu meer dan ooit bij de ouders, die zelf lange tijd
naar school zijn geweest en al enige tijd werken voordat ze ouder zijn
geworden. Op school en werk hebben ze zich dat doelgerichte, met de
nadruk op kennis en competenties, eigen gemaakt. Op school en werk leidt
dat natuurlijk tot succes. Maar in de opvoeding thuis is dat niet zozeer
het geval. Ook lijkt het er op dat ouders van nu alleen maar oog voor
het detail hebben. Ze stellen zichzelf vragen als hoe lang moeten ze hun
kinderen moeten laten huilen, of de computer wel goed voor ze is en of
ze wel lang genoeg huiswerk maken. De grote lijn lijken ze uit het oog
te zijn verloren.

Van jouw boek heb ik veel geleerd. Als ontwikkelingspsycholoog bied je
de lezer nieuwe inzichten, als filosoof lever je interessante dilemma's
en als ouder/grootouder laat je zien met welke concrete situaties je te
maken hebt gehad. Opvoeden tegenwoordig, zo schrijf jij, lijkt misschien
wel het meest op het werk van een meubelmaker. Er is ruim aandacht voor
het juiste materiaal waarmee gewerkt wordt, voor de juiste bewerking en
ouders willen net als de meubelmaker ervoor zorgen dat het juiste
product wordt afgemaakt. In het proces gaat het om precisie en controle
en chaos en verschil moeten zoveel mogelijk worden voorkomen. Maar
jijzelf hebt veel op met het idee dat opvoeden als het zorgen voor een
tuin is. Want als tuinman moet je de tuin beschermen en planten ruimte
geven waar nodig. Je moet er veel voor doen maar het onverwachte is hier
veel belangrijker. Net als de tuinman kun je als ouder niet alles
voorspellen en variatie en wanorde lijken er hier wel degelijk toe te
doen. Ik moet je eerlijk zeggen dat ik samen met mijn vrouw vaak in de
tuin werk. Ik weet (en mijn vrouw nog veel beter) dat tuinieren wel
degelijk doelgerichte kanten kent. Je moet soms dingen doen om een half
jaar later bepaalde resultaten te bereiken. Niet alles maar wel voor een
groot deel. Bovendien heb je hele verschillende tuinen. Natuurlijke
tuinen en tuinen waar precisie en controle wel van afspatten. Je hebt
Engelse tuinen en je hebt Franse tuinen. Kinderen waarvan de
ontwikkeling in gevaar komt om welke redenen dan ook, zullen ondersteund
moeten worden. Over hoe dat het beste kan en waar we rekening mee moeten
houden moeten we na blijven denken. Dan doen we ook in de tuin als
planten niet tot hun recht komen. Dat zullen we doelgericht moeten
blijven doen en ik neem aan dat je die gedachte deelt. Met het
vergelijken van opvoeden met houtbewerken aan de ene kant en tuinieren
aan de andere kant kan ik niet helemaal met jou meekomen. Misschien,
denk ik zelf, had je voor jouw boek veel beter de metafoor van het lange
en avontuurlijke reizen met het aankomen als doel kunnen nemen, zoals
Kavafis dat zo treffend in zijn beroemde gedicht `Ithaka' beschrijft. De
reis die vele jaren duurt maar rijk is aan wat je onderweg beleeft. Bij
opvoeden gaat het om het aankomen maar veel meer nog om die mooie reis.
Maar laat ik zelf ook de grote lijn van jouw boek in de gaten houden en
het punt dat je maakt is terecht.

Dank je wel,

Grote groet, -Harrie

Gopnik, A.(2016). \emph{The gardener and the carpenter. What the new
science of child development tells us about the relationship between
parents and children.} New York: Farrar, Strauss and Giroux, 320 pag.,
€22,16.

\hypertarget{brits-werelderfgoed}{%
\chapter*{BRITS WERELDERFGOED}\label{brits-werelderfgoed}}
\addcontentsline{toc}{chapter}{BRITS WERELDERFGOED}

Maart, 2017

Beste Helen Pearson,

Jij noemt de naoorlogse Britse cohortstudies (de vijf studies die er in
jouw land sinds 1946 zijn opgezet om de ontwikkeling van het leven van
gewone mensen van geboorte tot en met de dood te volgen) een juweel in
de kroon van de Britse wetenschap. In ieder geval maak jij in \emph{`The
Life Project. The extraordinary story of our ordinary lives'} wel heel
duidelijk hoe uitzonderlijk deze studies zijn, welke schat aan inzichten
en beleidsmaatregelen ze hebben opgeleverd en hoe ze duidelijke
antwoorden geven op belangrijke vragen als waarom sommige mensen in het
leven gelukkig, gezond en succesvol worden en dit voor anderen niet is
weggelegd. Inderdaad, buitengewone studies van grote groepen hele gewone
mensen. Veel wetenschappers hebben de levens van vijf generaties
zorgvuldig gevolgd en dat heeft geleid tot terabytes data, tientallen
boeken, duizenden artikelen, vriezers vol met DNA materiaal, dozen vol
met nagels, babytandjes en zelfs opslag van duizenden placenta's. Jij
bent wetenschapsjournalist en de eindredacteur van het bekende blad
\emph{`Nature'}. In 2010 las je voor het eerst over deze studies en in
2011 bezocht je een feestje van een van deze cohorten. Daarna was je
niet meer te houden. In de vijf jaar erna werk je aan en beschrijf je
heel zorgvuldig de geschiedenis van deze cohortstudies. Je bezoekt in
die tijd talloze bibliotheken, laboratoria en onderzoekkelders en
spreekt met genetici, economen, epidemiologen, sociale wetenschappers en
statistici. Jouw verhaal is een verhaal geworden van overleven, van
strijdvaardige en bijzondere wetenschappers die voortdurend hard vechten
om hun studie overeind te houden en die met hun geloof, verbondenheid,
eigenwijsheid, kennis, verbeelding en, toch ook wel, excentriciteit
ervoor zorgen dat deze studies doorgezet worden. Jouw boek is een
sociale en wetenschapsgeschiedenis tegelijkertijd en het lezen ervan,
moet ik jou zeggen, deed mij heel veel plezier, niet in de laatste
plaats vanwege jouw jaloersmakende schrijfstijl.

Jouw boek begint wanneer in 1946 een gek groepje wetenschappers alle
informatie gaat verzamelen over de geboorte van elke Britse baby die er
in een kille maartse week van dat jaar werd geboren en wanneer talloze
verpleegsters op de fiets het land doortrekken. Het leven van duizenden
van hen wordt sinds die eerste meting gevolgd en deze eerste Britse
cohort hield vorig jaar zijn zeventigste verjaardag. Vergelijkbare
cohorten worden daarna nogmaals uitgezet onder kinderen die in 1958 zijn
geboren, in 1970, begin jaren negentig en in 2000 (millennium kinderen).
In totaal worden er meer dan 70.000 kinderen op hun levenspad gevolgd.
In de eerste jaren is er uiteraard veel aandacht voor vraagstukken die
met de geboorte van kinderen te maken hebben. De eerste inzichten zijn
schokkend toen duidelijk werd hoe sterk het land door klassen was
verdeeld en de kindersterfte 70\% hoger was onder kinderen van de
werkende klasse. De inzichten leiden er in deze tijd toe dat
ondersteuning rond de geboorte in het aanbod van de net opgerichte
Nationale Health Service wordt opgenomen. Vanaf de jaren zestig wordt er
steeds meer over het onderwijs gesproken. Er zou sprake zijn van het
verlies van talent en dat kwam met name voor onder de arme kinderen. De
cohortstudies brengen deze onderwijsachterstanden duidelijk aan het
licht. Begin jaren zeventig wordt er nog volop gerookt in de westerse
samenleving en roken was ondertussen als een duidelijke risicofactor
voor ziekten en een verkorte levensduur gedetecteerd. Vanwege roken
stierven volwassenen eerder, maar wat betekent dat roken voor kinderen?
40\% van de zwangere vrouwen rookte begin jaren zeventig nog en de
cohortstudies laten in deze jaren overduidelijk zien dat dit roken
samenhangt met kindersterfte en lage geboortegewicht. Nadat de ergste
naoorlogse armoede is weggewerkt en de mensen het weer wat beter
krijgen, steken nieuwe ziekten de kop op. Obesitas en overgewicht nemen
met de welvaart toe en de cohortstudies laten zien dat gewicht onder de
verschillende cohorten tegelijkertijd sterk toeneemt. De leeftijden van
de cohortdeelnemers nemen verder toe en nieuwe problemen steken de kop
op. Zo wordt duidelijk gemaakt dat één op de drie ouders cognitieve
vaardigheden heeft die op het niveau van elf-jarigen liggen. Talloze
tests worden er onder deze volwassenen uitgezet om dit duidelijk te
maken. De data zijn zo sterk dat er ook inzichten kunnen worden gegeven
in de sociale mobiliteit en de intergenerationele overdracht ervan. De
sociaal economische toekomst van kinderen die in de jaren zeventig zijn
geboren is sterker aan die van hun ouders verbonden dan van kinderen die
in de jaren vijftig zijn geboren. Ook maken de cohortstudies duidelijk
hoe belangrijk de betrokkenheid van ouders in de eerste jaren is voor
een succesvol leven. Ondertussen zijn de eerste cohortdeelnemers senior
geworden en laat het zien met welke toename van stoornissen en ziekten
deze groep ouderen te maken heeft. In jouw boek maak ik, aandachtige
lezer, ook kennis met een groep hele bijzondere wetenschappers zoals
Douglas, Butler, Bynner, Wadsworth, Goldstein, Chalmers, Golding,
Kellner, Sullivan, Elliot, Goodman, Fitzesimon, Dezacourt en vele
anderen. In eerste instantie zijn het allemaal mannen, tegenwoordig zijn
alle cohortstudies in handen van vrouwen. In eerste instantie wordt de
verklaring voor onderliggende uitkomsten veelal gezocht in de omgeving
waarin mensen opgroeien, later spelen genetische elementen een
belangrijke rol. De studies zijn opgezet in een tijd dat er nog helemaal
geen gebruik van computers kan worden gemaakt. Daar kun je tegenwoordig
helemaal geen voorstelling van maken maar de gegevens werden toch
allemaal met de hand ingevoerd en op betrekkelijke eenvoudige manier
geanalyseerd. Jij laat ook zien waar er strijd wordt geleverd tussen
wetenschap en politiek maar ook tussen wetenschappelijke disciplines
onderling (sociale wetenschap bijvoorbeeld ten opzichte van medische
wetenschap).

Daar waar de cohortstudies doorgaan, wat jij van harte toejuicht,
eindigt jouw boek in 2015. Je laat zien hoe de vijf studies er dan
voorstaan, wat de verschillen tussen de vijf studies zijn en wie er voor
welke studie verantwoordelijk is. Ondertussen zijn er ook longitudinale
studies uitgezet in andere landen, zoals in Noorwegen, Denemarken,
Zweden en Amerika en ook in ons eigen land (bijvoorbeeld Generation R in
Rotterdam). Hier zijn ze dan misschien niet zo vroeg met deze levensloop
studies maar over de decennia heen zien we hier wel vergelijkbare
beleidsmaatregelen. Wat brachten de studies dan weer aan extra
informatie in en wat veranderde in Engeland wel en in de andere landen
niet? Dat haal ik toch niet helemaal uit jouw boek. Je maakt wel weer
expliciet duidelijk dat er geen land op dit gebied zoiets heeft
gepresteerd als het Verenigd Koninkrijk. Het Verenigd Koninkrijk kent 28
door de Unesco erkende werelderfgoederen. Daaronder vallen onder andere
Stonehedge, de Tower of London en de Canterbury Castle. Misschien zijn
die cohortstudies nog wel meer dan dat juweel in de Britse
wetenschappelijke kroon, waar jij het over hebt, en verdienen ze zelfs
een plaatsje in dat prachtig rijtje Brits werelderfgoederen. Daar waar
het Britse wereldrijk na de Tweede Wereldoorlog werd afgebroken, zijn
jullie in ieder geval op dit terrein wereldleider gebleven. Dat besef is
er helemaal niet maar jouw studie maakt duidelijk waar de Brit trots op
moet zijn en dat we ook wel wat breder tegen het cultureel erfgoed
kunnen aankijken.

Dank je wel.

Grote groet, -Harrie

Pearson, H.(2016). \emph{The Life Project. The extraordinary story of
our ordinary lives.} London: Penguin Books Ltd.~416 pagina's. €23,99

Jijzelf groeide op in de zestiger jaren, die gelukkige vijf minuten,
zoals je dat zo mooi omschrijft, tussen de pil en AIDS. Jij maakte deel
uit van die kritische generatie die nogal wat had op te merken over
ouders die zelf vaak in de crisistijd waren opgevoed. Nu stel je, onder
tussen zelf moeder en grootmoeder, dat jouw generatie (onze generatie)
het er niet heel veel beter vanaf heeft gebracht. Volgens jou is de
manier waarop we tegen opvoeden aankijken een verkeerde manier omdat die
niet past bij wetenschappelijke inzichten die laten zien dat we het de
kinderen veel meer op hun eigen manier moeten laten doen. Je hebt gelijk
dat het ouders van tegenwoordig veel te weinig gaat om variatie, risico
en innovatie en dat de opvoeding op die manier misschien wel te weinig
aansluit bij het evolutionaire doel van de kindertijd. Kindertijd is
vooral de tijd van exploratie, nieuwsgierigheid en spel en dat heb je
nodig voorafgaand aan de fase van exploiteren, verantwoordelijkheid en
werk. Wat er van elk kind wordt is onvoorspelbaar en uniek en het
resultaat van vreemde combinaties van genen, ervaringen, cultuur en
geluk. We moeten niet te snel een kind willen maken maar veelmeer
liefde, veiligheid en stabiliteit bieden waarin kinderen op hun manier
kunnen groeien. We kunnen nu één keer niet kinderen lerend maken, maar
we kunnen ze wel laten leren. Ouderschap zie jij terecht als een
belangrijk deel van de levenscyclus waarbij onze ouders ons het verleden
gaven en wij op onze beurt de toekomst aan onze kinderen doorgeven. Meer
dan bij welk ander levend soort ook zijn kinderen bij ons mensen lange
tijd afhankelijk van opvoeders. Zelfs in vroegere tijden waren ze niet
voor hun vijftiende zelfstandig. Kinderen waren daarbij niet alleen
afhankelijk van hun eigen ouders maar ook van `allo-ouders', het netwerk
van grootouders, ooms, tantes, neven, nichten en vrienden. Alleen komen
we als soort nergens en we ontwikkelen onszelf alleen in zo'n netwerk
van zorg en liefde. Lange tijd groeiden kinderen op in uitgebreide
families maar vandaag de dag is dat netwerk veel kleiner geworden. De
opvoedingstaak ligt nu meer dan ooit bij de ouders, die zelf lange tijd
naar school zijn geweest en al enige tijd werken voordat ze ouder zijn
geworden. Op school en werk hebben ze zich dat doelgerichte, met de
nadruk op kennis en competenties, eigen gemaakt. Op school en werk leidt
dat natuurlijk tot succes. Maar in de opvoeding thuis is dat niet zozeer
het geval. Ook lijkt het er op dat ouders van nu alleen maar oog voor
het detail hebben. Ze stellen zichzelf vragen als hoe lang moeten ze hun
kinderen moeten laten huilen, of de computer wel goed voor ze is en of
ze wel lang genoeg huiswerk maken. De grote lijn lijken ze uit het oog
te zijn verloren.

Van jouw boek heb ik veel geleerd. Als ontwikkelingspsycholoog bied je
de lezer nieuwe inzichten, als filosoof lever je interessante dilemma's
en als ouder/grootouder laat je zien met welke concrete situaties je te
maken hebt gehad. Opvoeden tegenwoordig, zo schrijf jij, lijkt misschien
wel het meest op het werk van een meubelmaker. Er is ruim aandacht voor
het juiste materiaal waarmee gewerkt wordt, voor de juiste bewerking en
ouders willen net als de meubelmaker ervoor zorgen dat het juiste
product wordt afgemaakt. In het proces gaat het om precisie en controle
en chaos en verschil moeten zoveel mogelijk worden voorkomen. Maar
jijzelf hebt veel op met het idee dat opvoeden als het zorgen voor een
tuin is. Want als tuinman moet je de tuin beschermen en planten ruimte
geven waar nodig. Je moet er veel voor doen maar het onverwachte is hier
veel belangrijker. Net als de tuinman kun je als ouder niet alles
voorspellen en variatie en wanorde lijken er hier wel degelijk toe te
doen. Ik moet je eerlijk zeggen dat ik samen met mijn vrouw vaak in de
tuin werk. Ik weet (en mijn vrouw nog veel beter) dat tuinieren wel
degelijk doelgerichte kanten kent. Je moet soms dingen doen om een half
jaar later bepaalde resultaten te bereiken. Niet alles maar wel voor een
groot deel. Bovendien heb je hele verschillende tuinen. Natuurlijke
tuinen en tuinen waar precisie en controle wel van afspatten. Je hebt
Engelse tuinen en je hebt Franse tuinen. Kinderen waarvan de
ontwikkeling in gevaar komt om welke redenen dan ook, zullen ondersteund
moeten worden. Over hoe dat het beste kan en waar we rekening mee moeten
houden moeten we na blijven denken. Dan doen we ook in de tuin als
planten niet tot hun recht komen. Dat zullen we doelgericht moeten
blijven doen en ik neem aan dat je die gedachte deelt. Met het
vergelijken van opvoeden met houtbewerken aan de ene kant en tuinieren
aan de andere kant kan ik niet helemaal met jou meekomen. Misschien,
denk ik zelf, had je voor jouw boek veel beter de metafoor van het lange
en avontuurlijke reizen met het aankomen als doel kunnen nemen, zoals
Kavafis dat zo treffend in zijn beroemde gedicht `Ithaka' beschrijft. De
reis die vele jaren duurt maar rijk is aan wat je onderweg beleeft. Bij
opvoeden gaat het om het aankomen maar veel meer nog om die mooie reis.
Maar laat ik zelf ook de grote lijn van jouw boek in de gaten houden en
het punt dat je maakt is terecht.

Dank je wel,

Grote groet, -Harrie

Gopnik, A.(2016). \emph{The gardener and the carpenter. What the new
science of child development tells us about the relationship between
parents and children.} New York: Farrar, Strauss and Giroux, 320 pag.,
€22,16.

\hypertarget{we-zullen-het-zelf-moeten-doen}{%
\chapter*{WE ZULLEN HET ZELF MOETEN
DOEN}\label{we-zullen-het-zelf-moeten-doen}}
\addcontentsline{toc}{chapter}{WE ZULLEN HET ZELF MOETEN DOEN}

Amsterdam, april 2017

Beste Paul Verhaeghe,

De drievuldigheid of triniteit die we van het Christendom kennen paste
Freud op de mens toe door hem te begrijpen vanuit id, ego en superego.
Freud zette de mens met beide benen op de grond. Het id is, volgens
Freud, de onbewuste laag van de persoonlijkheid, daar waar onze wensen
en verlangens zitten. Aan de andere kant zit het superego, het geweten,
het deel dat ons zegt wat mag en niet mag; een soort Japie als het ware,
de groene krekel van Pinokkio. Tussen id en superego zit ego dat
voortdurend bemiddelt tussen individuele wensen en verlangens en
maatschappelijke regels. Die drievuldige interactie tussen het
menselijke organisme en zijn omgeving zien we ook bij de Amerikaanse
socioloog George Herbert Mead. Hij had het over het zelf dat
communiceert met het ik, de pure vorm van het zelf, en het mij, de
sociale vorm van het zelf. Jijzelf bent psycholoog die na Freud en Mead
ook greep wil krijgen op de mens, maar nu die van honderd jaar later.
Jij vraagt je al langer af waarom er tegenwoordig zoveel mensen zijn die
tegen allerlei problemen aanlopen in het dagelijkse leven. Zij hebben
met existentiële vragen te maken die samenhangen met maatschappelijke
veranderingen die mensen ongelukkig maken. Anders dan bij Freud (waar
jij een kenner van bent) en anders dan bij Mead (waarop jouw werk
misschien wel veel mee lijkt) gaat het bij jou om identiteit, dat aan de
bovenkant bemiddelt met autoriteit en aan de onderkant met intimiteit.
Over identiteit en autoriteit schreef jij de afgelopen jaren
achtereenvolgens twee boeken: \emph{Identiteit} (Verhaege 2012) en
\emph{Autoriteit} (Verhaege 2015). Aan het derde boek,
\emph{Intimiteit}, werk je op dit moment.

Onlangs maakte je de tussenbalans op en sprak je in de Kohmstammlezing
over \emph{Identiteit, Autoriteit, Onderwijs} (Verhaege, 2017). Ik zat
bij die lezing ergens vooraan en luisterde vol interesse naar jouw
verhaal. Daar vertelde je nog maar weer eens dat onze identiteit door de
maatschappelijke veranderingen een andere invulling heeft gekregen.
Identiteit wordt, zeker vandaag de dag, nogal eens opgevat als iets dat
niet verandert en een harde kern heeft. Daarover denk jij terecht heel
anders. Identiteit is veel minder eigen dan wij vaak denken en verbonden
met een maatschappelijke omgeving waarbinnen wij ons ontwikkelen. Kan
identiteit zelf steeds verschillend zijn, het proces naar
identiteitsontwikkeling, waar je misschien wel meer over te zeggen hebt,
is steeds hetzelfde. Dat zie jij als een constructieproces dat over de
tijd invulling krijgt en waarbij tegelijk sprake is van identificatie en
separatie, van optrekken aan en afzetten tegen, van meekrijgen en zelf
bepalen. Identificatie (waar we veel over weten) verklaart vooral waarom
bepaalde mensen uit dezelfde omgeving en cultuur meer op elkaar lijken.
Separatie (waar we weinig over weten) laat de eigen en andere keuzes
zien. Deze twee wisselen elkaar niet over de tijd af maar zijn
voortdurend en altijd met elkaar in de weer. Omdat bij jou identiteit
iets relationeels is en jij het steeds hebt over verhoudingen tot de
ander, stel je identiteitsverhoudingen centraal. Voor jou zijn dat er
vier: de verhouding tot jezelf, het andere geslacht, de andere-gelijke
en de andere-autoriteit.\\
In je hele werk ben je bijzonder kritisch over de neoliberale
organisatie van onze samenleving met een economische ondertoon die zo
bepalend is voor wie we zijn. Met jouw kritiek ben ik het eens, maar of
dat door onze economische kijk op dingen komt vraag ik mij af. Je hebt
economen van diverse pluimage en volgens mij heeft het veelmeer met onze
eenzijdige instrumentele rationaliteit te maken, die inderdaad minder
oog heeft voor het gemeenschapsgevoel en het individu centraal stelt met
nadruk op competentie, concurrentie en eigen succes. In deze
hyperindividuele, instrumentele samenleving moet je het maken. Lukt dat
niet dan ben je een loser en voel je je schuldig. Natuurlijk kan niet
iedereen aan die hoge eisen voldoen, dat zie jij dagelijks. Maar er zijn
ook problemen waar je met identiteit alleen niet wegkomt en waarbij je
misschien wel een laag hoger moet kijken. Veel problemen vandaag de dag
hebben namelijk met gebrek aan gezag of autoriteit te maken, van
internationaal terrorisme tot en met ouders en leerkrachten die niet
meer goed weten hoe ze met hun kinderen om moeten gaan. Hier gaat het
jou niet zozeer om existentiële problemen maar meer om problemen die te
maken hebben met westerse waarden en normen. In jouw boeken en lezing
ontrafel je eerst wat het is en waarom we hier tegenwoordig problemen
mee hebben. Veel mensen denken dat die met de teloorgang van de westerse
samenleving samenhangen, de vreemdeling die in ons leven is gekomen en
het verdwijnen van een overzichtelijke samenleving. Veel mensen denken
ook dat het allemaal iets van de laatste tien tot twintig jaar is.
Terecht, denk ik, wijs jij er fijntjes op dat dat helemaal niet zo nieuw
is. Vanaf de zeventiende eeuw al zijn we de autoriteit van Kerk en
Koning in twijfel gaan trekken en is er een democratiseringsproces op
gang gekomen dat op bepaalde momenten maatschappelijk zichtbaar wordt.
Zo dachten we in de zestiger jaren dat we het maar beter zonder
autoriteit af konden. Werd er toen een teveel aan autoriteit ervaren,
nu, enkele decennia later, denken we weer dat er veel te weinig
autoriteit is. Autoriteit kent volgens jou geen natuurlijke bron maar
regelt, zoals je in het verlengde van Hannah Arendt schrijft, de
verhoudingen tussen mensen op grond van vrijwillige onderwerping
gebaseerd op een interne dwang. Bij jou gaat het daarbij om de
verhouding tot de ander. Hiermee weten we nog steeds niet goed raad.
Autoriteit verschilt zo fundamenteel van macht waarbij het gaat om iets
wat van buiten wordt opgelegd, waaraan je bent onderworpen en dat je kan
dwingen.

Het vrijemarktmodel dat ons cultureel op allerlei manieren omringt,
beïnvloedt het onderwijs natuurlijk sterk. Om succesvol te kunnen zijn
is hier de nadruk komen te liggen op competenties en vaardigheden en
gaat het om zelfmanagement en ondernemerschap. En natuurlijk krijgen we
nu de leerlingen die we gevraagd hebben en die vanuit zichzelf vragen
wat het onderwijs hen oplevert. Maar jij bent niet alleen kritisch over
de eenzijdige identiteitsontwikkeling waar het onderwijs aan bijdraagt.
Het onderwijs, eigenlijk de hele opvoeding, is ook nog eens verlegen
geworden ten aanzien van zijn eigen handelen. Ouders en leerkrachten
zijn onzeker geworden en voelen zich niet door het systeem gesteund.
Autoriteit moet volgens jou opnieuw gegrond worden en onderlinge
verhoudingen opnieuw gedefinieerd. Autoriteit kan niet meer van buiten
en bovenaf opgelegd worden maar moet van onderop tot stand komen. Jij
hebt veel op met autonome individuen die onderdeel uitmaken van een
netwerk waar ze zelf in geloven en die niet naar boven en beneden kijken
maar naar hoe anderen opzij reageren. Het onderwijs moet netwerken
rondom de leerling vormen, waar docenten en bestuurders, ouders en de
sociale omgeving deel van uitmaken. Scholen waar mensen naast elkaar
staan en naar elkaar verwijzen. Veel hiervan zul jij (zullen wij) in de
toekomst verder moeten uitwerken. Ook verdient de samenhang tussen
identiteit en autoriteit, en straks ook nog intimiteit, veel van jouw
aandacht. Maar je hebt gelijk dat we op zoek moeten naar nieuwe vormen
van sociale organisatie die bovendien volume krijgen op verschillende
niveaus van de samenleving. Maar hoe dit op schaal te brengen, vroeg is
mij wel direct af. Duidelijk werd het mij wel dat er veel is te doen,
dat drong die middag wel zachtjes tot mij door. Jouw boek `Autoriteit'
eindigde er al mee en jouw lezing sloot je er ook mee af; woorden die
mij heel erg aanspreken, waarmee je mij deelgenoot maakt van een lange,
interessante traditie, mij wakker schudt in het kerkbankje en mij aan
het werk zet: we zullen het zelf moeten doen.

Dank je wel.

Grote groet, -Harrie

Verhaege, P.(2017). \emph{Identiteit, Autoriteit, Onderwijs}.
Kohnstammlezing. Amsterdam: Vossiuspers UvA.

Verhaege, P. (2015). \emph{Autoriteit}. Amsterdam\textbar{}Antwerpen: De
Bezige Bij.

Verhaege, P. (2012). \emph{Identiteit}. Amsterdam: De Bezige Bij.

Jijzelf groeide op in de zestiger jaren, die gelukkige vijf minuten,
zoals je dat zo mooi omschrijft, tussen de pil en AIDS. Jij maakte deel
uit van die kritische generatie die nogal wat had op te merken over
ouders die zelf vaak in de crisistijd waren opgevoed. Nu stel je, onder
tussen zelf moeder en grootmoeder, dat jouw generatie (onze generatie)
het er niet heel veel beter vanaf heeft gebracht. Volgens jou is de
manier waarop we tegen opvoeden aankijken een verkeerde manier omdat die
niet past bij wetenschappelijke inzichten die laten zien dat we het de
kinderen veel meer op hun eigen manier moeten laten doen. Je hebt gelijk
dat het ouders van tegenwoordig veel te weinig gaat om variatie, risico
en innovatie en dat de opvoeding op die manier misschien wel te weinig
aansluit bij het evolutionaire doel van de kindertijd. Kindertijd is
vooral de tijd van exploratie, nieuwsgierigheid en spel en dat heb je
nodig voorafgaand aan de fase van exploiteren, verantwoordelijkheid en
werk. Wat er van elk kind wordt is onvoorspelbaar en uniek en het
resultaat van vreemde combinaties van genen, ervaringen, cultuur en
geluk. We moeten niet te snel een kind willen maken maar veelmeer
liefde, veiligheid en stabiliteit bieden waarin kinderen op hun manier
kunnen groeien. We kunnen nu één keer niet kinderen lerend maken, maar
we kunnen ze wel laten leren. Ouderschap zie jij terecht als een
belangrijk deel van de levenscyclus waarbij onze ouders ons het verleden
gaven en wij op onze beurt de toekomst aan onze kinderen doorgeven. Meer
dan bij welk ander levend soort ook zijn kinderen bij ons mensen lange
tijd afhankelijk van opvoeders. Zelfs in vroegere tijden waren ze niet
voor hun vijftiende zelfstandig. Kinderen waren daarbij niet alleen
afhankelijk van hun eigen ouders maar ook van `allo-ouders', het netwerk
van grootouders, ooms, tantes, neven, nichten en vrienden. Alleen komen
we als soort nergens en we ontwikkelen onszelf alleen in zo'n netwerk
van zorg en liefde. Lange tijd groeiden kinderen op in uitgebreide
families maar vandaag de dag is dat netwerk veel kleiner geworden. De
opvoedingstaak ligt nu meer dan ooit bij de ouders, die zelf lange tijd
naar school zijn geweest en al enige tijd werken voordat ze ouder zijn
geworden. Op school en werk hebben ze zich dat doelgerichte, met de
nadruk op kennis en competenties, eigen gemaakt. Op school en werk leidt
dat natuurlijk tot succes. Maar in de opvoeding thuis is dat niet zozeer
het geval. Ook lijkt het er op dat ouders van nu alleen maar oog voor
het detail hebben. Ze stellen zichzelf vragen als hoe lang moeten ze hun
kinderen moeten laten huilen, of de computer wel goed voor ze is en of
ze wel lang genoeg huiswerk maken. De grote lijn lijken ze uit het oog
te zijn verloren.

Van jouw boek heb ik veel geleerd. Als ontwikkelingspsycholoog bied je
de lezer nieuwe inzichten, als filosoof lever je interessante dilemma's
en als ouder/grootouder laat je zien met welke concrete situaties je te
maken hebt gehad. Opvoeden tegenwoordig, zo schrijf jij, lijkt misschien
wel het meest op het werk van een meubelmaker. Er is ruim aandacht voor
het juiste materiaal waarmee gewerkt wordt, voor de juiste bewerking en
ouders willen net als de meubelmaker ervoor zorgen dat het juiste
product wordt afgemaakt. In het proces gaat het om precisie en controle
en chaos en verschil moeten zoveel mogelijk worden voorkomen. Maar
jijzelf hebt veel op met het idee dat opvoeden als het zorgen voor een
tuin is. Want als tuinman moet je de tuin beschermen en planten ruimte
geven waar nodig. Je moet er veel voor doen maar het onverwachte is hier
veel belangrijker. Net als de tuinman kun je als ouder niet alles
voorspellen en variatie en wanorde lijken er hier wel degelijk toe te
doen. Ik moet je eerlijk zeggen dat ik samen met mijn vrouw vaak in de
tuin werk. Ik weet (en mijn vrouw nog veel beter) dat tuinieren wel
degelijk doelgerichte kanten kent. Je moet soms dingen doen om een half
jaar later bepaalde resultaten te bereiken. Niet alles maar wel voor een
groot deel. Bovendien heb je hele verschillende tuinen. Natuurlijke
tuinen en tuinen waar precisie en controle wel van afspatten. Je hebt
Engelse tuinen en je hebt Franse tuinen. Kinderen waarvan de
ontwikkeling in gevaar komt om welke redenen dan ook, zullen ondersteund
moeten worden. Over hoe dat het beste kan en waar we rekening mee moeten
houden moeten we na blijven denken. Dan doen we ook in de tuin als
planten niet tot hun recht komen. Dat zullen we doelgericht moeten
blijven doen en ik neem aan dat je die gedachte deelt. Met het
vergelijken van opvoeden met houtbewerken aan de ene kant en tuinieren
aan de andere kant kan ik niet helemaal met jou meekomen. Misschien,
denk ik zelf, had je voor jouw boek veel beter de metafoor van het lange
en avontuurlijke reizen met het aankomen als doel kunnen nemen, zoals
Kavafis dat zo treffend in zijn beroemde gedicht `Ithaka' beschrijft. De
reis die vele jaren duurt maar rijk is aan wat je onderweg beleeft. Bij
opvoeden gaat het om het aankomen maar veel meer nog om die mooie reis.
Maar laat ik zelf ook de grote lijn van jouw boek in de gaten houden en
het punt dat je maakt is terecht.

Dank je wel,

Grote groet, -Harrie

Gopnik, A.(2016). \emph{The gardener and the carpenter. What the new
science of child development tells us about the relationship between
parents and children.} New York: Farrar, Strauss and Giroux, 320 pag.,
€22,16.

\bibliography{book.bib,packages.bib}


\end{document}
